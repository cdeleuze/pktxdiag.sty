\documentclass[a4paper]{ltxdoc}
\usepackage{a4wide}
\usepackage[utf8]{inputenc}
\usepackage{pktxdiag}
\usepackage{pgfmanual}
\usepackage{calc}
\usepackage{subfig}
% Copyright 2006 by Till Tantau
%
% This file may be distributed and/or modified
%
% 1. under the LaTeX Project Public License and/or
% 2. under the GNU Free Documentation License.
%
% See the file doc/generic/pgf/licenses/LICENSE for more details.

% $Header: /cvsroot/pgf/pgf/doc/generic/pgf/macros/pgfmanual-en-macros.tex,v 1.79 2013/12/20 15:22:35 tantau Exp $


\providecommand\href[2]{\texttt{#1}}
\providecommand\hypertarget[2]{\texttt{#1}}
\providecommand\hyperlink[2]{\texttt{#1}}


\colorlet{examplefill}{yellow!80!black}
\definecolor{graphicbackground}{rgb}{0.96,0.96,0.8}
\definecolor{codebackground}{rgb}{0.9,0.9,1}

\newenvironment{pgfmanualentry}{\list{}{\leftmargin=2em\itemindent-\leftmargin\def\makelabel##1{\hss##1}}}{\endlist}
\newcommand\pgfmanualentryheadline[1]{\itemsep=0pt\parskip=0pt{\raggedright\item\strut{#1}\par}\topsep=0pt}
\newcommand\pgfmanualbody{\parskip3pt}

\let\origtexttt=\texttt
\def\texttt#1{{\def\textunderscore{\char`\_}\def\textbraceleft{\char`\{}\def\textbraceright{\char`\}}\origtexttt{#1}}}
\def\exclamationmarktext{!}
\def\atmarktext{@}

{
  \catcode`\|=12
  \gdef\pgfmanualnormalbar{|}
  \catcode`\|=13
  \AtBeginDocument{\gdef|{\ifmmode\pgfmanualnormalbar\else\expandafter\verb\expandafter|\fi}}
}



\newenvironment{pgflayout}[1]{
  \begin{pgfmanualentry}
    \pgfmanualentryheadline{%
      \pgfmanualpdflabel{#1}{}%
      \texttt{\string\pgfpagesuselayout\char`\{\declare{#1}\char`\}}\oarg{options}%
    }
    \index{#1@\protect\texttt{#1} layout}%
    \index{Page layouts!#1@\protect\texttt{#1}}%
    \pgfmanualbody
}
{
  \end{pgfmanualentry}
}


\newenvironment{command}[1]{
  \begin{pgfmanualentry}
    \extractcommand#1\@@
    \pgfmanualbody
}
{
  \end{pgfmanualentry}
}

\makeatletter

\def\includeluadocumentationof#1{
  \directlua{require 'pgf.manual.DocumentParser'}
  \directlua{pgf.manual.DocumentParser.include '#1'}
}

\newenvironment{luageneric}[4]{
  \pgfmanualentry
    \pgfmanualentryheadline{#4 \texttt{#1\declare{#2}}#3}
    \index{#2@\protect\texttt{#2} (Lua)}%
    \def\temp{#1}
    \ifx\temp\pgfutil@empty\else
      \index{#1@\protect\texttt{#1}!#2@\protect\texttt{#2} (Lua)}%
    \fi
  \pgfmanualbody
}{\endpgfmanualentry}

\newenvironment{luatable}[3]{
  \medskip
  \luageneric{#1}{#2}{ (declared in \texttt{#3})}{\textbf{Lua table}}
}{\endluageneric}

\newenvironment{luafield}[1]{
  \pgfmanualentry
    \pgfmanualentryheadline{Field \texttt{\declare{#1}}}
  \pgfmanualbody
}{\endpgfmanualentry}


\newenvironment{lualibrary}[1]{
  \pgfmanualentry
  \pgfmanualentryheadline{%
    \pgfmanualpdflabel{#1}{}%
    \textbf{Graph Drawing Library} \texttt{\declare{#1}}%
  }
    \index{#1@\protect\texttt{#1} graph drawing library}%
    \index{Libraries!#1@\protect\texttt{#1}}%
    \index{Graph drawing libraries!#1@\protect\texttt{#1}}%
    \vskip.25em
    {\ttfamily\char`\\usegdlibrary\char`\{\declare{#1}\char`\}\space\space \char`\%\space\space  \LaTeX\space and plain \TeX}\\
    {\ttfamily\char`\\usegdlibrary[\declare{#1}]\space \char`\%\space\space Con\TeX t}\smallskip\par
    \pgfmanualbody
}{\endpgfmanualentry}

\newenvironment{luadeclare}[4]{
  \pgfmanualentry
  \def\manual@temp@default{#3}%
  \def\manual@temp@initial{#4}%
  \def\manual@temp@{#3#4}%
  \pgfmanualentryheadline{%
    \pgfmanualpdflabel{#1}{}%
    {\ttfamily/graph
      drawing/\declare{#1}\opt{=}}\opt{#2}\hfill%
    \ifx\manual@temp@\pgfutil@empty\else%
    (\ifx\manual@temp@default\pgfutil@empty\else%
    default {\ttfamily #3}\ifx\manual@temp@initial\pgfutil@empty\else, \fi%
    \fi%
    \ifx\manual@temp@initial\pgfutil@empty\else%
    initially {\ttfamily #4}%
    \fi%
    )\fi%
  }%
  \index{#1@\protect\texttt{#1} key}%
  \pgfmanualbody
  \gdef\myname{#1}%
%  \keyalias{tikz}
%  \keyalias{tikz/graphs}
}{\endpgfmanualentry}

\newenvironment{luadeclarestyle}[4]{
  \pgfmanualentry
  \def\manual@temp@para{#2}%
  \def\manual@temp@default{#3}%
  \def\manual@temp@initial{#4}%
  \def\manual@temp@{#3#4}%
  \pgfmanualentryheadline{%
    \pgfmanualpdflabel{#1}{}%
    {\ttfamily/graph drawing/\declare{#1}}\ifx\manual@temp@para\pgfutil@empty\else\opt{\texttt=}\opt{#2}\fi\hfill%
    (style\ifx\manual@temp@\pgfutil@empty\else, %
    \ifx\manual@temp@default\pgfutil@empty\else%
    default {\ttfamily #3}\ifx\manual@temp@initial\pgfutil@empty\else, \fi%
    \fi%
    \ifx\manual@temp@initial\pgfutil@empty\else%
    initially {\ttfamily #4}%
    \fi%
    \fi)%
  }%
  \index{#1@\protect\texttt{#1} key}%
  \pgfmanualbody%
  \gdef\myname{#1}%
%  \keyalias{tikz}
%  \keyalias{tikz/graphs}
}{\endpgfmanualentry}

\newenvironment{luanamespace}[2]{
  \luageneric{#1}{#2}{}{\textbf{Lua namespace}}
}{\endluageneric}

\newenvironment{luafiledescription}[1]{}{}

\newenvironment{luacommand}[4]{
  \hypertarget{pgf/lua/#1}{\luageneric{#2}{#3}{\texttt{(#4)}}{\texttt{function}}}
}{\endluageneric}

\newenvironment{luaparameters}{\par\emph{Parameters:}%
  \parametercount=0\relax%
  \let\item=\parameteritem%
  \let\list=\restorelist%
}
{\par
}

\newenvironment{luareturns}{\par\emph{Returns:}%
  \parametercount=0\relax%
  \let\item=\parameteritem%
  \let\list=\restorelist%
}
{\par
}

\newcount\parametercount

\newenvironment{parameterdescription}{\unskip%
  \parametercount=0\relax%
  \let\item=\parameteritem%
  \let\list=\restorelist%
}
{\par
}
\let\saveditemcommand=\item
\let\savedlistcommand=\list
\def\denselist#1#2{\savedlistcommand{#1}{#2}\parskip0pt\itemsep0pt}
\def\restorelist{\let\item=\saveditemcommand\denselist}
\def\parameteritem{\pgfutil@ifnextchar[\parameteritem@{}}%}
\def\parameteritem@[#1]{\advance\parametercount by1\relax\hskip0.15em plus 1em\emph{\the\parametercount.}\kern1ex\def\test{#1}\ifx\test\pgfutil@empty\else#1\kern.5em\fi}

\newenvironment{commandlist}[1]{%
  \begin{pgfmanualentry}
  \foreach \xx in {#1} {%
    \expandafter\extractcommand\xx\@@
  }%
  \pgfmanualbody
}{%
  \end{pgfmanualentry}
}%

% \begin{internallist}[register]{\pgf@xa}
% \end{internallist}
%
% \begin{internallist}[register]{\pgf@xa,\pgf@xb}
% \end{internallist}
\newenvironment{internallist}[2][register]{%
  \begin{pgfmanualentry}
  \foreach \xx in {#2} {%
    \expandafter\extractinternalcommand\expandafter{\xx}{#1}%
  }%
  \pgfmanualbody
}{%
  \end{pgfmanualentry}
}%
\def\extractinternalcommand#1#2{%
  \removeats{#1}%
  \pgfmanualentryheadline{%
    \pgfmanualpdflabel{\textbackslash\strippedat}{}%
    Internal #2 \declare{\texttt{\string#1}}}%
  \index{Internals!\strippedat @\protect\myprintocmmand{\strippedat}}%
  \index{\strippedat @\protect\myprintocmmand{\strippedat}}%
}

%% MW: START MATH MACROS
\def\mvar#1{{\ifmmode\textrm{\textit{#1}}\else\rmfamily\textit{#1}\fi}}

\makeatletter

\def\extractmathfunctionname#1{\extractmathfunctionname@#1(,)\tmpa\tmpb}
\def\extractmathfunctionname@#1(#2)#3\tmpb{\def\mathname{#1}}

\makeatother
  
\newenvironment{math-function}[1]{
  \def\mathdefaultname{#1}
  \extractmathfunctionname{#1}
  \edef\mathurl{{math:\mathname}}\expandafter\hypertarget\expandafter{\mathurl}{}%
  \begin{pgfmanualentry}
    \pgfmanualentryheadline{\texttt{#1}}%
    \index{\mathname @\protect\texttt{\mathname} math function}%
    \index{Math functions!\mathname @\protect\texttt{\mathname}}%
    \pgfmanualbody
}
{
  \end{pgfmanualentry}
}

\def\pgfmanualemptytext{}
\def\pgfmanualvbarvbar{\char`\|\char`\|}

\newenvironment{math-operator}[4][]{%
  \begin{pgfmanualentry}
  \csname math#3operator\endcsname{#2}{#4}
  \def\mathtest{#4}%
  \ifx\mathtest\pgfmanualemptytext%
    \def\mathtype{(#3 operator)}
  \else%
    \def\mathtype{(#3 operator; uses the \texttt{#4} function)}
  \fi%
  \pgfmanualentryheadline{\mathexample\hfill\mathtype}%
  \def\mathtest{#1}%
  \ifx\mathtest\pgfmanualemptytext%
    \index{#2@\protect\texttt{#2} #3 math operator}%  
    \index{Math operators!#2@\protect\texttt{#2}}%
  \fi%
  \pgfmanualbody
}
{\end{pgfmanualentry}}

\newenvironment{math-operators}[5][]{%
  \begin{pgfmanualentry}
  \csname math#4operator\endcsname{#2}{#3}
  \def\mathtest{#5}%
  \ifx\mathtest\pgfmanualemptytext%
    \def\mathtype{(#4 operators)}
  \else%
    \def\mathtype{(#4 operators; use the \texttt{#5} function)}
  \fi%
  \pgfmanualentryheadline{\mathexample\hfill\mathtype}%
  \def\mathtest{#1}%
  \ifx\mathtest\pgfmanualemptytext%
    \index{#2#3@\protect\texttt{#2\protect\ #3} #4 math operators}% 
    \index{Math operators!#2#3@\protect\texttt{#2\protect\ #3}}%
  \fi%
  \pgfmanualbody
}
{\end{pgfmanualentry}}

\def\mathinfixoperator#1#2{%
  \def\mathoperator{\texttt{#1}}%
  \def\mathexample{\mvar{x}\space\texttt{#1}\space\mvar{y}}%
}

\def\mathprefixoperator#1#2{%
  \def\mathoperator{\texttt{#1}}%
  \def\mathexample{\texttt{#1}\mvar{x}}%
}

\def\mathpostfixoperator#1#2{%
  \def\mathoperator{\texttt{#1}}
  \def\mathexample{\mvar{x}\texttt{#1}}%
}

\def\mathgroupoperator#1#2{%
  \def\mathoperator{\texttt{#1\ #2}}%
  \def\mathexample{\texttt{#1}\mvar{x}\texttt{#2}}%
}

\expandafter\let\csname matharray accessoperator\endcsname=\mathgroupoperator
\expandafter\let\csname matharrayoperator\endcsname=\mathgroupoperator

\def\mathconditionaloperator#1#2{%
  \def\mathoperator{#1\space#2}
  \def\mathexample{\mvar{x}\ \texttt{#1}\ \mvar{y}\ {\texttt{#2}}\ \mvar{z}}
}

\newcommand\mathcommand[1][\mathdefaultname]{%
  \expandafter\makemathcommand#1(\empty)\stop%
  \expandafter\extractcommand\mathcommandname\@@%
  \medskip
}
\makeatletter

\def\makemathcommand#1(#2)#3\stop{%
  \expandafter\def\expandafter\mathcommandname\expandafter{\csname pgfmath#1\endcsname}%
  \ifx#2\empty%
  \else%
    \@makemathcommand#2,\stop,
  \fi}
\def\@makemathcommand#1,{%
  \ifx#1\stop%
  \else%
    \expandafter\def\expandafter\mathcommandname\expandafter{\mathcommandname{\ttfamily\char`\{#1\char`\}}}%
    \expandafter\@makemathcommand%
  \fi}
\makeatother

\def\calcname{\textsc{calc}}

\newenvironment{math-keyword}[1]{
  \extracttikzmathkeyword#1@
  \begin{pgfmanualentry}
    \pgfmanualentryheadline{\texttt{\color{red}\mathname}\mathrest}%
    \index{\mathname @\protect\texttt{\mathname} tikz math function}%
    \index{TikZ math functions!\mathname @\protect\texttt{\mathname}}%
    \pgfmanualbody
}
{
  \end{pgfmanualentry}
}

\def\extracttikzmathkeyword#1#2@{%
  \def\mathname{#1}%
  \def\mathrest{#2}%
}

%% MW: END MATH MACROS


\def\extractcommand#1#2\@@{%
  \removeats{#1}%
  \pgfmanualentryheadline{%
    \pgfmanualpdflabel{\textbackslash\strippedat}{}%
    \declare{\texttt{\string#1}}#2%
  }%
  \index{\strippedat @\protect\myprintocmmand{\strippedat}}
}

\def\luaextractcommand#1#2\relax{%
  \declare{\texttt{\string#1}}#2\par%
%  \removeats{#1}%
 % \index{\strippedat @\protect\myprintocmmand{\strippedat}}
 % \pgfmanualpdflabel{\textbackslash\strippedat}{}%
}


% \begin{environment}{{name}\marg{arguments}}
\renewenvironment{environment}[1]{
  \begin{pgfmanualentry}
    \extractenvironement#1\@@
    \pgfmanualbody
}
{
  \end{pgfmanualentry}
}

\def\extractenvironement#1#2\@@{%
  \pgfmanualentryheadline{%
    \pgfmanualpdflabel{#1}{}%
    {\ttfamily\char`\\begin\char`\{\declare{#1}\char`\}}#2%
  }%
  \pgfmanualentryheadline{{\ttfamily\ \ }\meta{environment contents}}%
  \pgfmanualentryheadline{{\ttfamily\char`\\end\char`\{\declare{#1}\char`\}}}%
  \index{#1@\protect\texttt{#1} environment}%
  \index{Environments!#1@\protect\texttt{#1}}
}


\newenvironment{plainenvironment}[1]{
  \begin{pgfmanualentry}
    \extractplainenvironement#1\@@
    \pgfmanualbody
}
{
  \end{pgfmanualentry}
}

\def\extractplainenvironement#1#2\@@{%
  \pgfmanualentryheadline{{\ttfamily\declare{\char`\\#1}}#2}%
  \pgfmanualentryheadline{{\ttfamily\ \ }\meta{environment contents}}%
  \pgfmanualentryheadline{{\ttfamily\declare{\char`\\end#1}}}%
  \index{#1@\protect\texttt{#1} environment}%
  \index{Environments!#1@\protect\texttt{#1}}%
}


\newenvironment{contextenvironment}[1]{
  \begin{pgfmanualentry}
    \extractcontextenvironement#1\@@
    \pgfmanualbody
}
{
  \end{pgfmanualentry}
}

\def\extractcontextenvironement#1#2\@@{%
  \pgfmanualentryheadline{{\ttfamily\declare{\char`\\start#1}}#2}%
  \pgfmanualentryheadline{{\ttfamily\ \ }\meta{environment contents}}%
  \pgfmanualentryheadline{{\ttfamily\declare{\char`\\stop#1}}}%
  \index{#1@\protect\texttt{#1} environment}%
  \index{Environments!#1@\protect\texttt{#1}}}


\newenvironment{shape}[1]{
  \begin{pgfmanualentry}
    \pgfmanualentryheadline{%
      \pgfmanualpdflabel{#1}{}%
      \textbf{Shape} {\ttfamily\declare{#1}}%
    }%
    \index{#1@\protect\texttt{#1} shape}%
    \index{Shapes!#1@\protect\texttt{#1}}
    \pgfmanualbody
}
{
  \end{pgfmanualentry}
}

\newenvironment{pictype}[2]{
  \begin{pgfmanualentry}
    \pgfmanualentryheadline{%
      \pgfmanualpdflabel{#1}{}%
      \textbf{Pic type} {\ttfamily\declare{#1}#2}%
    }%
    \index{#1@\protect\texttt{#1} pic type}%
    \index{Pic Types!#1@\protect\texttt{#1}}
    \pgfmanualbody
}
{
  \end{pgfmanualentry}
}

\newenvironment{shading}[1]{
  \begin{pgfmanualentry}
    \pgfmanualentryheadline{%
      \pgfmanualpdflabel{#1}{}%
      \textbf{Shading} {\ttfamily\declare{#1}}}%
    \index{#1@\protect\texttt{#1} shading}%
    \index{Shadings!#1@\protect\texttt{#1}}
    \pgfmanualbody
}
{
  \end{pgfmanualentry}
}


\newenvironment{graph}[1]{
  \begin{pgfmanualentry}
    \pgfmanualentryheadline{%
      \pgfmanualpdflabel{#1}{}%
      \textbf{Graph} {\ttfamily\declare{#1}}}%
    \index{#1@\protect\texttt{#1} graph}%
    \index{Graphs!#1@\protect\texttt{#1}}
    \pgfmanualbody
}
{
  \end{pgfmanualentry}
}

\newenvironment{gdalgorithm}[2]{
  \begin{pgfmanualentry}
    \pgfmanualentryheadline{%
      \pgfmanualpdflabel{#1}{}%
      \textbf{Layout} {\ttfamily/graph drawing/\declare{#1}\opt{=}}\opt{\meta{options}}}%
    \index{#1@\protect\texttt{#1} layout}%
    \index{Layouts!#1@\protect\texttt{#1}}%
    \foreach \algo in {#2}
    {\edef\marshal{\noexpand\index{#2@\noexpand\protect\noexpand\texttt{#2} algorithm}}\marshal}%
    \index{Graph drawing layouts!#1@\protect\texttt{#1}}
    \item{\small alias {\ttfamily/tikz/#1}}\par
    \item{\small alias {\ttfamily/tikz/graphs/#1}}\par
    \item{\small Employs {\ttfamily algorithm=#2}}\par
    \pgfmanualbody
}
{
  \end{pgfmanualentry}
}

\newenvironment{dataformat}[1]{
  \begin{pgfmanualentry}
    \pgfmanualentryheadline{%
      \pgfmanualpdflabel{#1}{}%
      \textbf{Format} {\ttfamily\declare{#1}}}%
    \index{#1@\protect\texttt{#1} format}%
    \index{Formats!#1@\protect\texttt{#1}}
    \pgfmanualbody
}
{
  \end{pgfmanualentry}
}

\newenvironment{stylesheet}[1]{
  \begin{pgfmanualentry}
    \pgfmanualentryheadline{%
      \pgfmanualpdflabel{#1}{}%
      \textbf{Style sheet} {\ttfamily\declare{#1}}}%
    \index{#1@\protect\texttt{#1} style sheet}%
    \index{Style sheets!#1@\protect\texttt{#1}}
    \pgfmanualbody
}
{
  \end{pgfmanualentry}
}

\newenvironment{handler}[1]{
  \begin{pgfmanualentry}
    \extracthandler#1\@nil%
    \pgfmanualbody
}
{
  \end{pgfmanualentry}
}

\def\gobble#1{}
\def\extracthandler#1#2\@nil{%
  \pgfmanualentryheadline{%
    \pgfmanualpdflabel{/handlers/#1}{}%
    \textbf{Key handler} \meta{key}{\ttfamily/\declare{#1}}#2}%
  \index{\gobble#1@\protect\texttt{#1} handler}%
  \index{Key handlers!#1@\protect\texttt{#1}}
}


\makeatletter


\newenvironment{stylekey}[1]{
  \begin{pgfmanualentry}
    \def\extrakeytext{style, }
    \extractkey#1\@nil%
    \pgfmanualbody
}
{
  \end{pgfmanualentry}
}

\def\choicesep{$\vert$}%
\def\choicearg#1{\texttt{#1}}

\newif\iffirstchoice

% \mchoice{choice1,choice2,choice3}
\newcommand\mchoice[1]{%
  \begingroup
  \firstchoicetrue
  \foreach \mchoice@ in {#1} {%
    \iffirstchoice
      \global\firstchoicefalse
    \else
      \choicesep
    \fi
    \choicearg{\mchoice@}%
  }%
  \endgroup
}%

% \begin{key}{/path/x=value}
% \begin{key}{/path/x=value (initially XXX)}
% \begin{key}{/path/x=value (default XXX)}
\newenvironment{key}[1]{
  \begin{pgfmanualentry}
    \def\extrakeytext{}
    %\def\altpath{\emph{\color{gray}or}}%
    \extractkey#1\@nil%
    \pgfmanualbody
}
{
  \end{pgfmanualentry}
}

% \insertpathifneeded{a key}{/pgf} -> assign mykey={/pgf/a key}
% \insertpathifneeded{/tikz/a key}{/pgf} -> assign mykey={/tikz/a key}
%
% #1: the key
% #2: a default path (or empty)
\def\insertpathifneeded#1#2{%
  \def\insertpathifneeded@@{#2}%
  \ifx\insertpathifneeded@@\empty
    \def\mykey{#1}%
  \else
    \insertpathifneeded@#2\@nil
    \ifpgfutil@in@
      \def\mykey{#2/#1}%
    \else
      \def\mykey{#1}%
    \fi
  \fi
}%
\def\insertpathifneeded@#1#2\@nil{%
  \def\insertpathifneeded@@{#1}%
  \def\insertpathifneeded@@@{/}%
  \ifx\insertpathifneeded@@\insertpathifneeded@@@
    \pgfutil@in@true
  \else
    \pgfutil@in@false
  \fi
}%

% \begin{keylist}[default path]
%   {/path/option 1=value,/path/option 2=value2}
% \end{keylist}
\newenvironment{keylist}[2][]{%
  \begin{pgfmanualentry}
    \def\extrakeytext{}%
  \foreach \xx in {#2} {%
    \expandafter\insertpathifneeded\expandafter{\xx}{#1}%
    \expandafter\extractkey\mykey\@nil%
  }%
  \pgfmanualbody
}{%
  \end{pgfmanualentry}
}%

\def\extractkey#1\@nil{%
  \pgfutil@in@={#1}%
  \ifpgfutil@in@%
    \extractkeyequal#1\@nil
  \else%
    \pgfutil@in@{(initial}{#1}%
    \ifpgfutil@in@%
      \extractequalinitial#1\@nil%
    \else
      \pgfmanualentryheadline{%
      \def\mykey{#1}%
      \def\mypath{}%
      \gdef\myname{}%
      \firsttimetrue%
      \pgfmanualdecomposecount=0\relax%
      \decompose#1/\nil%
        {\ttfamily\declare{#1}}\hfill(\extrakeytext no value)}%
    \fi
  \fi%
}

\def\extractkeyequal#1=#2\@nil{%
  \pgfutil@in@{(default}{#2}%
  \ifpgfutil@in@%
    \extractdefault{#1}#2\@nil%
  \else%
    \pgfutil@in@{(initial}{#2}%
    \ifpgfutil@in@%
      \extractinitial{#1}#2\@nil%
    \else
      \pgfmanualentryheadline{%
        \def\mykey{#1}%
        \def\mypath{}%
        \gdef\myname{}%
        \firsttimetrue%
        \pgfmanualdecomposecount=0\relax%
        \decompose#1/\nil%
        {\ttfamily\declare{#1}=}#2\hfill(\extrakeytext no default)}%
    \fi%
  \fi%
}

\def\extractdefault#1#2(default #3)\@nil{%
  \pgfmanualentryheadline{%
    \def\mykey{#1}%
    \def\mypath{}%
    \gdef\myname{}%
    \firsttimetrue%
    \pgfmanualdecomposecount=0\relax%
    \decompose#1/\nil%
    {\ttfamily\declare{#1}\opt{=}}\opt{#2}\hfill (\extrakeytext default {\ttfamily#3})}%
}

\def\extractinitial#1#2(initially #3)\@nil{%
  \pgfmanualentryheadline{%
    \def\mykey{#1}%
    \def\mypath{}%
    \gdef\myname{}%
    \firsttimetrue%
    \pgfmanualdecomposecount=0\relax%
    \decompose#1/\nil%
    {\ttfamily\declare{#1}=}#2\hfill (\extrakeytext no default, initially {\ttfamily#3})}%
}

\def\extractequalinitial#1 (initially #2)\@nil{%
  \pgfmanualentryheadline{%
    \def\mykey{#1}%
    \def\mypath{}%
    \gdef\myname{}%
    \firsttimetrue%
    \pgfmanualdecomposecount=0\relax%
    \decompose#1/\nil%
    {\ttfamily\declare{#1}}\hfill (\extrakeytext initially {\ttfamily#2})}%
}

% Introduces a key alias '/#1/<name of current key>'
% to be used inside of \begin{key} ... \end{key}
\def\keyalias#1{\vspace{-3pt}\item{\small alias {\ttfamily/#1/\myname}}\vspace{-2pt}\par
  \pgfmanualpdflabel{/#1/\myname}{}%
}

\newif\iffirsttime
\newcount\pgfmanualdecomposecount

\makeatother

\def\decompose/#1/#2\nil{%
  \def\test{#2}%
  \ifx\test\empty%
    % aha.
    \index{#1@\protect\texttt{#1} key}%
    \index{\mypath#1@\protect\texttt{#1}}%
    \gdef\myname{#1}%
    \pgfmanualpdflabel{#1}{}
  \else%
    \advance\pgfmanualdecomposecount by1\relax%
    \ifnum\pgfmanualdecomposecount>2\relax%
      \decomposetoodeep#1/#2\nil%
    \else%
      \iffirsttime%
        \begingroup%  
          % also make a pdf link anchor with full key path.
          \def\hyperlabelwithoutslash##1/\nil{%
            \pgfmanualpdflabel{##1}{}%
          }%
          \hyperlabelwithoutslash/#1/#2\nil%
        \endgroup%
        \def\mypath{#1@\protect\texttt{/#1/}!}%
        \firsttimefalse%
      \else%
        \expandafter\def\expandafter\mypath\expandafter{\mypath#1@\protect\texttt{#1/}!}%
      \fi%
      \def\firsttime{}%
      \decompose/#2\nil%
    \fi%
  \fi%
}

\def\decomposetoodeep#1/#2/\nil{%
  % avoid too-deep nesting in index
  \index{#1/#2@\protect\texttt{#1/#2} key}%
  \index{\mypath#1/#2@\protect\texttt{#1/#2}}%
  \decomposefindlast/#1/#2/\nil%
}
\makeatletter
\def\decomposefindlast/#1/#2\nil{%
  \def\test{#2}%
  \ifx\test\pgfutil@empty%
    \gdef\myname{#1}%
  \else%
    \decomposefindlast/#2\nil%
  \fi%
}
\makeatother
\def\indexkey#1{%
  \def\mypath{}%
  \decompose#1/\nil%
}

\newenvironment{predefinedmethod}[1]{
  \begin{pgfmanualentry}
    \extractpredefinedmethod#1\@nil
    \pgfmanualbody
}
{
  \end{pgfmanualentry}
}
\def\extractpredefinedmethod#1(#2)\@nil{%
  \pgfmanualentryheadline{%
    \pgfmanualpdflabel{#1}{}%
    Method \declare{\ttfamily #1}\texttt(#2\texttt) \hfill(predefined for all classes)}
  \index{#1@\protect\texttt{#1} method}%
  \index{Methods!#1@\protect\texttt{#1}}
}


\newenvironment{ooclass}[1]{
  \begin{pgfmanualentry}
    \def\currentclass{#1}
    \pgfmanualentryheadline{%
      \pgfmanualpdflabel{#1}{}%
      \textbf{Class} \declare{\texttt{#1}}}
    \index{#1@\protect\texttt{#1} class}%
    \index{Class #1@Class \protect\texttt{#1}}%
    \index{Classes!#1@\protect\texttt{#1}}
    \pgfmanualbody
}
{
  \end{pgfmanualentry}
}

\newenvironment{method}[1]{
  \begin{pgfmanualentry}
    \extractmethod#1\@nil
    \pgfmanualbody
}
{
  \end{pgfmanualentry}
}
\def\extractmethod#1(#2)\@nil{%
  \def\test{#1}
  \ifx\test\currentclass
    \pgfmanualentryheadline{%
      \pgfmanualpdflabel{#1}{}%
      Constructor \declare{\ttfamily #1}\texttt(#2\texttt)}
  \else
    \pgfmanualentryheadline{%
      \pgfmanualpdflabel{#1}{}%
      Method \declare{\ttfamily #1}\texttt(#2\texttt)}
  \fi
  \index{#1@\protect\texttt{#1} method}%
  \index{Methods!#1@\protect\texttt{#1}}
  \index{Class \currentclass!#1@\protect\texttt{#1}}%
}

\newenvironment{attribute}[1]{
  \begin{pgfmanualentry}
    \extractattribute#1\@nil
    \pgfmanualbody
}
{
  \end{pgfmanualentry}
}
\def\extractattribute#1=#2;\@nil{%
  \def\test{#2}%
  \ifx\test\@empty
    \pgfmanualentryheadline{%
      \pgfmanualpdflabel{#1}{}%
      Private attribute \declare{\ttfamily #1} \hfill (initially empty)}
  \else
    \pgfmanualentryheadline{%
      \pgfmanualpdflabel{#1}{}%
      Private attribute \declare{\ttfamily #1} \hfill (initially {\ttfamily #2})}
  \fi
  \index{#1@\protect\texttt{#1} attribute}%
  \index{Attributes!#1@\protect\texttt{#1}}
  \index{Class \currentclass!#1@\protect\texttt{#1}}%
}



\newenvironment{predefinednode}[1]{
  \begin{pgfmanualentry}
    \pgfmanualentryheadline{%
      \pgfmanualpdflabel{#1}{}%
      \textbf{Predefined node} {\ttfamily\declare{#1}}}%
    \index{#1@\protect\texttt{#1} node}%
    \index{Predefined node!#1@\protect\texttt{#1}}
    \pgfmanualbody
}
{
  \end{pgfmanualentry}
}

\newenvironment{coordinatesystem}[1]{
  \begin{pgfmanualentry}
    \pgfmanualentryheadline{%
      \pgfmanualpdflabel{#1}{}%
      \textbf{Coordinate system} {\ttfamily\declare{#1}}}%
    \index{#1@\protect\texttt{#1} coordinate system}%
    \index{Coordinate systems!#1@\protect\texttt{#1}}
    \pgfmanualbody
}
{
  \end{pgfmanualentry}
}

\newenvironment{snake}[1]{
  \begin{pgfmanualentry}
    \pgfmanualentryheadline{\textbf{Snake} {\ttfamily\declare{#1}}}%
    \index{#1@\protect\texttt{#1} snake}%
    \index{Snakes!#1@\protect\texttt{#1}}
    \pgfmanualbody
}
{
  \end{pgfmanualentry}
}

\newenvironment{decoration}[1]{
  \begin{pgfmanualentry}
    \pgfmanualentryheadline{\textbf{Decoration} {\ttfamily\declare{#1}}}%
    \index{#1@\protect\texttt{#1} decoration}%
    \index{Decorations!#1@\protect\texttt{#1}}
    \pgfmanualbody
}
{
  \end{pgfmanualentry}
}


\def\pgfmanualbar{\char`\|}
\makeatletter
\newenvironment{pathoperation}[3][]{
  \begin{pgfmanualentry}
    \def\pgfmanualtest{#1}%
    \pgfmanualentryheadline{%
      \ifx\pgfmanualtest\@empty%
        \pgfmanualpdflabel{#2}{}%
      \fi%
      \textcolor{gray}{{\ttfamily\char`\\path}\
        \ \dots}
      \declare{\texttt{#2}}#3\ \textcolor{gray}{\dots\texttt{;}}}%
    \ifx\pgfmanualtest\@empty%
      \index{#2@\protect\texttt{#2} path operation}%
      \index{Path operations!#2@\protect\texttt{#2}}%
    \fi%
    \pgfmanualbody
}
{
  \end{pgfmanualentry}
}
\newenvironment{datavisualizationoperation}[3][]{
  \begin{pgfmanualentry}
    \def\pgfmanualtest{#1}%
    \pgfmanualentryheadline{%
      \ifx\pgfmanualtest\@empty%
        \pgfmanualpdflabel{#2}{}%
      \fi%
      \textcolor{gray}{{\ttfamily\char`\\datavisualization}\
        \ \dots}
      \declare{\texttt{#2}}#3\ \textcolor{gray}{\dots\texttt{;}}}%
    \ifx\pgfmanualtest\@empty%
      \index{#2@\protect\texttt{#2} (data visualization)}%
      \index{Data visualization!#2@\protect\texttt{#2}}%
    \fi%
    \pgfmanualbody
}
{
  \end{pgfmanualentry}
}
\makeatother

\def\doublebs{\texttt{\char`\\\char`\\}}


\newenvironment{package}[1]{
  \begin{pgfmanualentry}
    \pgfmanualentryheadline{%
      \pgfmanualpdflabel{#1}{}%
      {\ttfamily\char`\\usepackage\char`\{\declare{#1}\char`\}\space\space \char`\%\space\space  \LaTeX}}
    \index{#1@\protect\texttt{#1} package}%
    \index{Packages and files!#1@\protect\texttt{#1}}%
    \pgfmanualentryheadline{{\ttfamily\char`\\input \declare{#1}.tex\space\space\space \char`\%\space\space  plain \TeX}}
    \pgfmanualentryheadline{{\ttfamily\char`\\usemodule[\declare{#1}]\space\space \char`\%\space\space  Con\TeX t}}
    \pgfmanualbody
}
{
  \end{pgfmanualentry}
}


\newenvironment{pgfmodule}[1]{
  \begin{pgfmanualentry}
    \pgfmanualentryheadline{%
      \pgfmanualpdflabel{#1}{}%
      {\ttfamily\char`\\usepgfmodule\char`\{\declare{#1}\char`\}\space\space\space
        \char`\%\space\space  \LaTeX\space and plain \TeX\space and pure pgf}}
    \index{#1@\protect\texttt{#1} module}%
    \index{Modules!#1@\protect\texttt{#1}}%
    \pgfmanualentryheadline{{\ttfamily\char`\\usepgfmodule[\declare{#1}]\space\space \char`\%\space\space  Con\TeX t\space and pure pgf}}
    \pgfmanualbody
}
{
  \end{pgfmanualentry}
}

\newenvironment{pgflibrary}[1]{
  \begin{pgfmanualentry}
    \pgfmanualentryheadline{%
      \pgfmanualpdflabel{#1}{}%
      \textbf{\tikzname\ Library} \texttt{\declare{#1}}}
    \index{#1@\protect\texttt{#1} library}%
    \index{Libraries!#1@\protect\texttt{#1}}%
    \vskip.25em%
    {{\ttfamily\char`\\usepgflibrary\char`\{\declare{#1}\char`\}\space\space\space
        \char`\%\space\space  \LaTeX\space and plain \TeX\space and pure pgf}}\\
    {{\ttfamily\char`\\usepgflibrary[\declare{#1}]\space\space \char`\%\space\space  Con\TeX t\space and pure pgf}}\\
    {{\ttfamily\char`\\usetikzlibrary\char`\{\declare{#1}\char`\}\space\space
        \char`\%\space\space  \LaTeX\space and plain \TeX\space when using \tikzname}}\\
    {{\ttfamily\char`\\usetikzlibrary[\declare{#1}]\space
        \char`\%\space\space  Con\TeX t\space when using \tikzname}}\\[.5em]
    \pgfmanualbody
}
{
  \end{pgfmanualentry}
}

\newenvironment{purepgflibrary}[1]{
  \begin{pgfmanualentry}
    \pgfmanualentryheadline{%
      \pgfmanualpdflabel{#1}{}%
      \textbf{{\small PGF} Library} \texttt{\declare{#1}}}
    \index{#1@\protect\texttt{#1} library}%
    \index{Libraries!#1@\protect\texttt{#1}}%
    \vskip.25em%
    {{\ttfamily\char`\\usepgflibrary\char`\{\declare{#1}\char`\}\space\space\space
        \char`\%\space\space  \LaTeX\space and plain \TeX}}\\
    {{\ttfamily\char`\\usepgflibrary[\declare{#1}]\space\space \char`\%\space\space  Con\TeX t}}\\[.5em]
    \pgfmanualbody
}
{
  \end{pgfmanualentry}
}

\newenvironment{tikzlibrary}[1]{
  \begin{pgfmanualentry}
    \pgfmanualentryheadline{%
      \pgfmanualpdflabel{#1}{}%
      \textbf{\tikzname\ Library} \texttt{\declare{#1}}}
    \index{#1@\protect\texttt{#1} library}%
    \index{Libraries!#1@\protect\texttt{#1}}%
    \vskip.25em%
    {{\ttfamily\char`\\usetikzlibrary\char`\{\declare{#1}\char`\}\space\space \char`\%\space\space  \LaTeX\space and plain \TeX}}\\
    {{\ttfamily\char`\\usetikzlibrary[\declare{#1}]\space \char`\%\space\space Con\TeX t}}\\[.5em]
    \pgfmanualbody
}
{
  \end{pgfmanualentry}
}



\newenvironment{filedescription}[1]{
  \begin{pgfmanualentry}
    \pgfmanualentryheadline{File {\ttfamily\declare{#1}}}%
    \index{#1@\protect\texttt{#1} file}%
    \index{Packages and files!#1@\protect\texttt{#1}}%
    \pgfmanualbody
}
{
  \end{pgfmanualentry}
}


\newenvironment{packageoption}[1]{
  \begin{pgfmanualentry}
    \pgfmanualentryheadline{{\ttfamily\char`\\usepackage[\declare{#1}]\char`\{pgf\char`\}}}
    \index{#1@\protect\texttt{#1} package option}%
    \index{Package options for \textsc{pgf}!#1@\protect\texttt{#1}}%
    \pgfmanualbody
}
{
  \end{pgfmanualentry}
}



\newcommand\opt[1]{{\color{black!50!green}#1}}
\newcommand\ooarg[1]{{\ttfamily[}\meta{#1}{\ttfamily]}}

\def\opt{\afterassignment\pgfmanualopt\let\next=}
\def\pgfmanualopt{\ifx\next\bgroup\bgroup\color{black!50!green}\else{\color{black!50!green}\next}\fi}



\def\beamer{\textsc{beamer}}
\def\pdf{\textsc{pdf}}
\def\eps{\texttt{eps}}
\def\pgfname{\textsc{pgf}}
\def\tikzname{Ti\emph{k}Z}
\def\pstricks{\textsc{pstricks}}
\def\prosper{\textsc{prosper}}
\def\seminar{\textsc{seminar}}
\def\texpower{\textsc{texpower}}
\def\foils{\textsc{foils}}

{
  \makeatletter
  \global\let\myempty=\@empty
  \global\let\mygobble=\@gobble
  \catcode`\@=12
  \gdef\getridofats#1@#2\relax{%
    \def\getridtest{#2}%
    \ifx\getridtest\myempty%
      \expandafter\def\expandafter\strippedat\expandafter{\strippedat#1}
    \else%
      \expandafter\def\expandafter\strippedat\expandafter{\strippedat#1\protect\printanat}
      \getridofats#2\relax%
    \fi%
  }

  \gdef\removeats#1{%
    \let\strippedat\myempty%
    \edef\strippedtext{\stripcommand#1}%
    \expandafter\getridofats\strippedtext @\relax%
  }
  
  \gdef\stripcommand#1{\expandafter\mygobble\string#1}
}

\def\printanat{\char`\@}

\def\declare{\afterassignment\pgfmanualdeclare\let\next=}
\def\pgfmanualdeclare{\ifx\next\bgroup\bgroup\color{red!75!black}\else{\color{red!75!black}\next}\fi}


\let\textoken=\command
\let\endtextoken=\endcommand

\def\myprintocmmand#1{\texttt{\char`\\#1}}

\def\example{\par\smallskip\noindent\textit{Example: }}
\def\themeauthor{\par\smallskip\noindent\textit{Theme author: }}


\def\indexoption#1{%
  \index{#1@\protect\texttt{#1} option}%
  \index{Graphic options and styles!#1@\protect\texttt{#1}}%
}

\def\itemcalendaroption#1{\item \declare{\texttt{#1}}%
  \index{#1@\protect\texttt{#1} date test}%
  \index{Date tests!#1@\protect\texttt{#1}}%
}



\def\class#1{\list{}{\leftmargin=2em\itemindent-\leftmargin\def\makelabel##1{\hss##1}}%
\extractclass#1@\par\topsep=0pt}
\def\endclass{\endlist}
\def\extractclass#1#2@{%
\item{{{\ttfamily\char`\\documentclass}#2{\ttfamily\char`\{\declare{#1}\char`\}}}}%
  \index{#1@\protect\texttt{#1} class}%
  \index{Classes!#1@\protect\texttt{#1}}}

\def\partname{Part}

\makeatletter
\def\index@prologue{\section*{Index}\addcontentsline{toc}{section}{Index}
  This index only contains automatically generated entries. A good
  index should also contain carefully selected keywords. This index is
  not a good index.
  \bigskip
}
\c@IndexColumns=2
  \def\theindex{\@restonecoltrue
    \columnseprule \z@  \columnsep 29\p@
    \twocolumn[\index@prologue]%
       \parindent -30pt
       \columnsep 15pt
       \parskip 0pt plus 1pt
       \leftskip 30pt
       \rightskip 0pt plus 2cm
       \small
       \def\@idxitem{\par}%
    \let\item\@idxitem \ignorespaces}
  \def\endtheindex{\onecolumn}
\def\noindexing{\let\index=\@gobble}


\newenvironment{arrowtipsimple}[1]{
  \begin{pgfmanualentry}
    \pgfmanualentryheadline{\textbf{Arrow Tip Kind} {\ttfamily#1}}
    \index{#1@\protect\texttt{#1} arrow tip}%
    \index{Arrow tips!#1@\protect\texttt{#1}}%
    \def\currentarrowtype{#1}
    \pgfmanualbody}
{
  \end{pgfmanualentry}
}

\newenvironment{arrowtip}[4]{
  \begin{pgfmanualentry}
    \pgfmanualentryheadline{\textbf{Arrow Tip Kind} {\ttfamily#1}}
    \index{#1@\protect\texttt{#1} arrow tip}%
    \index{Arrow tips!#1@\protect\texttt{#1}}%
    \pgfmanualbody
    \def\currentarrowtype{#1}
    \begin{minipage}[t]{10.25cm}
      #2
    \end{minipage}\hskip5mm\begin{minipage}[t]{4.75cm}
      \leavevmode\vskip-2em
    \tikz{
      \draw [black!50,line width=5mm,-{#1[#3,color=black]}] (-4,0) -- (0,0);
      \foreach \action in {#4}
      { \expandafter\processaction\action\relax }
    }
    \end{minipage}\par\smallskip
  }
{
  \end{pgfmanualentry}
}

\newenvironment{arrowcap}[5]{
  \begin{pgfmanualentry}
    \pgfmanualentryheadline{\textbf{Arrow Tip Kind} {\ttfamily#1}}
    \index{#1@\protect\texttt{#1} arrow tip}%
    \index{Arrow tips!#1@\protect\texttt{#1}}%
    \pgfmanualbody
    \def\currentarrowtype{#1}
    \begin{minipage}[t]{10.25cm}
      #2
    \end{minipage}\hskip5mm\begin{minipage}[t]{4.75cm}
      \leavevmode\vskip-2em
    \tikz{
      \path [tips, line width=10mm,-{#1[#3,color=black]}] (-4,0) -- (0,0);
      \draw [line width=10mm,black!50] (-3,0) -- (#5,0);
      \foreach \action in {#4}
      { \expandafter\processaction\action\relax }
    }
    \end{minipage}\par\smallskip
  }
{
  \end{pgfmanualentry}
}

\def\processaction#1=#2\relax{
  \expandafter\let\expandafter\pgf@temp\csname manual@action@#1\endcsname
  \ifx\pgf@temp\relax\else
    \pgf@temp#2/0/\relax
  \fi
}
\def\manual@action@length#1/#2/#3\relax{%
  \draw [red,|<->|,semithick,xshift=#2] ([yshift=4pt]current bounding
  box.north -| -#1,0) coordinate (last length) -- node
  [above=-2pt] {|length|} ++(#1,0);
}
\def\manual@action@width#1/#2/#3\relax{%
  \draw [overlay, red,|<->|,semithick] (.5,-#1/2) -- node [below,sloped] {|width|} (.5,#1/2);
}
\def\manual@action@inset#1/#2/#3\relax{%
  \draw [red,|<->|,semithick,xshift=#2] ([yshift=-4pt]current bounding
  box.south -| last length) -- node [below] {|inset|} ++(#1,0);
}

\newenvironment{arrowexamples}
{\begin{tabbing}
    \hbox to \dimexpr\linewidth-5.5cm\relax{\emph{Appearance of the below at line width} \hfil} \= 
     \hbox to 1.9cm{\emph{0.4pt}\hfil} \= \hbox to 2cm{\emph{0.8pt}\hfil} \= \emph{1.6pt} \\
  }
{\end{tabbing}\vskip-1em}

\newenvironment{arrowcapexamples}
{\begin{tabbing}
    \hbox to \dimexpr\linewidth-5.5cm\relax{\emph{Appearance of the below at line width} \hfil} \= 
     \hbox to 1.9cm{\emph{1ex}\hfil} \= \hbox to 2cm{\emph{1em}\hfil} \\
  }
{\end{tabbing}\vskip-1em}

\def\arrowcapexample#1[#2]{\def\temp{#1}\ifx\temp\pgfutil@empty\arrowcapexample@\currentarrowtype[{#2}]\else\arrowcapexample@#1[{#2}]\fi}
\def\arrowcapexample@#1[#2]{%
  {\sfcode`\.1000\small\texttt{#1[#2]}} \>
  \kern-.5ex\tikz [baseline,>={#1[#2]}] \draw [line
  width=1ex,->] (0,.5ex) -- (2em,.5ex);  \>
  \kern-.5em\tikz [baseline,>={#1[#2]}] \draw [line
  width=1em,->] (0,.5ex) -- (2em,.5ex);  \\
}

\def\arrowexample#1[#2]{\def\temp{#1}\ifx\temp\pgfutil@empty\arrowexample@\currentarrowtype[{#2}]\else\arrowexample@#1[{#2}]\fi}
\def\arrowexample@#1[#2]{%
  {\sfcode`\.1000\small\texttt{#1[#2]}} \>
  \tikz [baseline,>={#1[#2]}] \draw [line
  width=0.4pt,->] (0,.5ex) -- (2em,.5ex); thin \>
  \tikz [baseline,>={#1[#2]}] \draw [line
  width=0.8pt,->] (0,.5ex) -- (2em,.5ex); \textbf{thick} \>
  \tikz [baseline,>={#1[#2]}] \draw [line
  width=1.6pt,->] (0,.5ex) -- (3em,.5ex); \\
}
\def\arrowexampledup[#1]{\arrowexample[{#1] \currentarrowtype[}]}
\def\arrowexampledupdot[#1]{\arrowexample[{#1] . \currentarrowtype[}]}

\def\arrowexampledouble#1[#2]{\def\temp{#1}\ifx\temp\pgfutil@empty\arrowexampledouble@\currentarrowtype[{#2}]\else\arrowexampledouble@#1[{#2}]\fi}
\def\arrowexampledouble@#1[#2]{%
  {\sfcode`\.1000\small\texttt{#1[#2]} on double line} \>
  \tikz [baseline,>={#1[#2]}]
    \draw [double equal sign distance,line width=0.4pt,->] (0,.5ex) -- (2em,.5ex); thin \>
  \tikz [baseline,>={#1[#2]}]
    \draw [double equal sign distance,line width=0.8pt,->] (0,.5ex) -- (2em,.5ex); \textbf{thick} \>
  \tikz [baseline,>={#1[#2]}]
    \draw [double equal sign distance, line width=1.6pt,->] (0,.5ex) -- (3em,.5ex); \\
}



\newcommand\symarrow[1]{%
  \index{#1@\protect\texttt{#1} arrow tip}%
  \index{Arrow tips!#1@\protect\texttt{#1}}%
  \texttt{#1}& yields thick  
  \begin{tikzpicture}[arrows={#1-#1},thick,baseline]
    \useasboundingbox (-1mm,-0.5ex) rectangle (1.1cm,2ex);
    \fill [black!15] (1cm,-.5ex) rectangle (1.1cm,1.5ex) (-1mm,-.5ex) rectangle (0mm,1.5ex) ;
    \draw (0pt,.5ex) -- (1cm,.5ex);
  \end{tikzpicture} and thin
  \begin{tikzpicture}[arrows={#1-#1},thin,baseline]
    \useasboundingbox (-1mm,-0.5ex) rectangle (1.1cm,2ex);
    \fill [black!15] (1cm,-.5ex) rectangle (1.1cm,1.5ex) (-1mm,-.5ex) rectangle (0mm,1.5ex) ;
    \draw (0pt,.5ex) -- (1cm,.5ex);
  \end{tikzpicture}
}
\newcommand\symarrowdouble[1]{%
  \index{#1@\protect\texttt{#1} arrow tip}%
  \index{Arrow tips!#1@\protect\texttt{#1}}%
  \texttt{#1}& yields thick  
  \begin{tikzpicture}[arrows={#1-#1},thick,baseline]
    \useasboundingbox (-1mm,-0.5ex) rectangle (1.1cm,2ex);
    \fill [black!15] (1cm,-.5ex) rectangle (1.1cm,1.5ex) (-1mm,-.5ex) rectangle (0mm,1.5ex) ;
    \draw (0pt,.5ex) -- (1cm,.5ex);
  \end{tikzpicture}
  and thin
  \begin{tikzpicture}[arrows={#1-#1},thin,baseline]
    \useasboundingbox (-1mm,-0.5ex) rectangle (1.1cm,2ex);
    \fill [black!15] (1cm,-.5ex) rectangle (1.1cm,1.5ex) (-1mm,-.5ex) rectangle (0mm,1.5ex) ;
    \draw (0pt,.5ex) -- (1cm,.5ex);
  \end{tikzpicture}, double 
  \begin{tikzpicture}[arrows={#1-#1},thick,baseline]
    \useasboundingbox (-1mm,-0.5ex) rectangle (1.1cm,2ex);
    \fill [black!15] (1cm,-.5ex) rectangle (1.1cm,1.5ex) (-1mm,-.5ex) rectangle (0mm,1.5ex) ;
    \draw[double,double equal sign distance] (0pt,.5ex) -- (1cm,.5ex);
  \end{tikzpicture} and 
  \begin{tikzpicture}[arrows={#1-#1},thin,baseline]
    \useasboundingbox (-1mm,-0.5ex) rectangle (1.1cm,2ex);
    \fill [black!15] (1cm,-.5ex) rectangle (1.1cm,1.5ex) (-1mm,-.5ex) rectangle (0mm,1.5ex) ;
    \draw[double,double equal sign distance] (0pt,.5ex) -- (1cm,.5ex);
  \end{tikzpicture}
}

\newcommand\sarrow[2]{%
  \index{#1@\protect\texttt{#1} arrow tip}%
  \index{Arrow tips!#1@\protect\texttt{#1}}%
  \index{#2@\protect\texttt{#2} arrow tip}%
  \index{Arrow tips!#2@\protect\texttt{#2}}%
  \texttt{#1-#2}& yields thick  
  \begin{tikzpicture}[arrows={#1-#2},thick,baseline]
    \useasboundingbox (-1mm,-0.5ex) rectangle (1.1cm,2ex);
    \fill [black!15] (1cm,-.5ex) rectangle (1.1cm,1.5ex) (-1mm,-.5ex) rectangle (0mm,1.5ex) ;
    \draw (0pt,.5ex) -- (1cm,.5ex);
  \end{tikzpicture} and thin
  \begin{tikzpicture}[arrows={#1-#2},thin,baseline]
    \useasboundingbox (-1mm,-0.5ex) rectangle (1.1cm,2ex);
    \fill [black!15] (1cm,-.5ex) rectangle (1.1cm,1.5ex) (-1mm,-.5ex) rectangle (0mm,1.5ex) ;
    \draw (0pt,.5ex) -- (1cm,.5ex);
  \end{tikzpicture}
}

\newcommand\sarrowdouble[2]{%
  \index{#1@\protect\texttt{#1} arrow tip}%
  \index{Arrow tips!#1@\protect\texttt{#1}}%
  \index{#2@\protect\texttt{#2} arrow tip}%
  \index{Arrow tips!#2@\protect\texttt{#2}}%
  \texttt{#1-#2}& yields thick  
  \begin{tikzpicture}[arrows={#1-#2},thick,baseline]
    \useasboundingbox (-1mm,-0.5ex) rectangle (1.1cm,2ex);
    \fill [black!15] (1cm,-.5ex) rectangle (1.1cm,1.5ex) (-1mm,-.5ex) rectangle (0mm,1.5ex) ;
    \draw (0pt,.5ex) -- (1cm,.5ex);
  \end{tikzpicture} and thin
  \begin{tikzpicture}[arrows={#1-#2},thin,baseline]
    \useasboundingbox (-1mm,-0.5ex) rectangle (1.1cm,2ex);
    \fill [black!15] (1cm,-.5ex) rectangle (1.1cm,1.5ex) (-1mm,-.5ex) rectangle (0mm,1.5ex) ;
    \draw (0pt,.5ex) -- (1cm,.5ex);
  \end{tikzpicture}, double 
  \begin{tikzpicture}[arrows={#1-#2},thick,baseline]
    \useasboundingbox (-1mm,-0.5ex) rectangle (1.1cm,2ex);
    \fill [black!15] (1cm,-.5ex) rectangle (1.1cm,1.5ex) (-1mm,-.5ex) rectangle (0mm,1.5ex) ;
    \draw[double,double equal sign distance] (0pt,.5ex) -- (1cm,.5ex);
  \end{tikzpicture} and 
  \begin{tikzpicture}[arrows={#1-#2},thin,baseline]
    \useasboundingbox (-1mm,-0.5ex) rectangle (1.1cm,2ex);
    \fill [black!15] (1cm,-.5ex) rectangle (1.1cm,1.5ex) (-1mm,-.5ex) rectangle (0mm,1.5ex) ;
    \draw[double,double equal sign distance] (0pt,.5ex) -- (1cm,.5ex);
  \end{tikzpicture}
}

\newcommand\carrow[1]{%
  \index{#1@\protect\texttt{#1} arrow tip}%
  \index{Arrow tips!#1@\protect\texttt{#1}}%
  \texttt{#1}& yields for line width 1ex
  \begin{tikzpicture}[arrows={#1-#1},line width=1ex,baseline]
    \useasboundingbox (-1mm,-0.5ex) rectangle (1.6cm,2ex);
    \fill [black!15] (1.5cm,-.5ex) rectangle (1.6cm,1.5ex) (-1mm,-.5ex) rectangle (0mm,1.5ex) ;
    \draw (0pt,.5ex) -- (1.5cm,.5ex);
  \end{tikzpicture}
}
\def\myvbar{\char`\|}
\newcommand\plotmarkentry[1]{%
  \index{#1@\protect\texttt{#1} plot mark}%
  \index{Plot marks!#1@\protect\texttt{#1}}
  \texttt{\char`\\pgfuseplotmark\char`\{\declare{#1}\char`\}} &
  \tikz\draw[color=black!25] plot[mark=#1,mark options={fill=examplefill,draw=black}] coordinates{(0,0) (.5,0.2) (1,0) (1.5,0.2)};\\
}
\newcommand\plotmarkentrytikz[1]{%
  \index{#1@\protect\texttt{#1} plot mark}%
  \index{Plot marks!#1@\protect\texttt{#1}}
  \texttt{mark=\declare{#1}} & \tikz\draw[color=black!25]
  plot[mark=#1,mark options={fill=examplefill,draw=black}] 
    coordinates {(0,0) (.5,0.2) (1,0) (1.5,0.2)};\\
}



\ifx\scantokens\@undefined
  \PackageError{pgfmanual-macros}{You need to use extended latex
    (elatex) or (pdfelatex) to process this document}{}
\fi

\begingroup
\catcode`|=0
\catcode`[= 1
\catcode`]=2
\catcode`\{=12
\catcode `\}=12
\catcode`\\=12 |gdef|find@example#1\end{codeexample}[|endofcodeexample[#1]]
|endgroup

% define \returntospace.
%
% It should define NEWLINE as {}, spaces and tabs as \space.
\begingroup
\catcode`\^=7
\catcode`\^^M=13
\catcode`\^^I=13
\catcode`\ =13%
\gdef\returntospace{\catcode`\ =13\def {\space}\catcode`\^^I=13\def^^I{\space}}
\gdef\showreturn{\show^^M}
\endgroup

\begingroup
\catcode`\%=13
\catcode`\^^M=13
\gdef\commenthandler{\catcode`\%=13\def%{\@gobble@till@return}}
\gdef\@gobble@till@return#1^^M{}
\gdef\@gobble@till@return@ignore#1^^M{\ignorespaces}
\gdef\typesetcomment{\catcode`\%=13\def%{\@typeset@till@return}}
\gdef\@typeset@till@return#1^^M{{\def%{\char`\%}\textsl{\char`\%#1}}\par}
\endgroup

% Define tab-implementation functions
%   \codeexample@tabinit@replacementchars@
% and
%   \codeexample@tabinit@catcode@
%
% They should ONLY be used in case that tab replacement is active.
%
% This here is merely a preparation step.
%
% Idea:
% \codeexample@tabinit@catcode@ will make TAB active
% and
% \codeexample@tabinit@replacementchars@ will insert as many spaces as
% /codeexample/tabsize contains.
{
\catcode`\^^I=13
% ATTENTION: do NOT use tabs in these definitions!!
\gdef\codeexample@tabinit@replacementchars@{%
 \begingroup
 \count0=\pgfkeysvalueof{/codeexample/tabsize}\relax
 \toks0={}%
 \loop
 \ifnum\count0>0
  \advance\count0 by-1
  \toks0=\expandafter{\the\toks0\ }%
 \repeat
 \xdef\codeexample@tabinit@replacementchars@@{\the\toks0}%
 \endgroup
 \let^^I=\codeexample@tabinit@replacementchars@@
}%
\gdef\codeexample@tabinit@catcode@{\catcode`\^^I=13}%
}%

% Called after any options have been set. It assigns
%   \codeexample@tabinit@catcode
% and
%   \codeexample@tabinit@replacementchars
% which are used inside of 
%\begin{codeexample}
% ...
%\end{codeexample}
%
% \codeexample@tabinit@catcode  is either \relax or it makes tab
% active.
%
% \codeexample@tabinit@replacementchars is either \relax or it inserts
% a proper replacement sequence for tabs (as many spaces as
% configured)
\def\codeexample@tabinit{%
  \ifnum\pgfkeysvalueof{/codeexample/tabsize}=0\relax
    \let\codeexample@tabinit@replacementchars=\relax
    \let\codeexample@tabinit@catcode=\relax
  \else
    \let\codeexample@tabinit@catcode=\codeexample@tabinit@catcode@
    \let\codeexample@tabinit@replacementchars=\codeexample@tabinit@replacementchars@
  \fi
}

\newif\ifpgfmanualtikzsyntaxhilighting

\pgfqkeys{/codeexample}{%
  width/.code=  {\setlength\codeexamplewidth{#1}},
  graphic/.code=  {\colorlet{graphicbackground}{#1}},
  code/.code=  {\colorlet{codebackground}{#1}},
  execute code/.is if=code@execute,
  code only/.code=  {\code@executefalse},
  pre/.store in=\code@pre,
  post/.store in=\code@post,
  % #1 is the *complete* environment contents as it shall be
  % typeset. In particular, the catcodes are NOT the normal ones.
  typeset listing/.code=  {#1},
  render instead/.store in=\code@render,
  vbox/.code=  {\def\code@pre{\vbox\bgroup\setlength{\hsize}{\linewidth-6pt}}\def\code@post{\egroup}},
  ignorespaces/.code=  {\let\@gobble@till@return=\@gobble@till@return@ignore},
  leave comments/.code=  {\def\code@catcode@hook{\catcode`\%=12}\let\commenthandler=\relax\let\typesetcomment=\relax},
  tabsize/.initial=0,% FIXME : this here is merely used for indentation. It is just a TAB REPLACEMENT.
  every codeexample/.style={width=4cm+7pt, tikz syntax=true},
  from file/.code={\codeexamplefromfiletrue\def\codeexamplesource{#1}},
  tikz syntax/.is if=pgfmanualtikzsyntaxhilighting,
}


\newread\examplesource


% Opening, reading and closing the results file

\def\opensource#1{
  \immediate\openin\examplesource=#1
}
\def\do@codeexamplefromfile{%
  \immediate\openin\examplesource\expandafter{\codeexamplesource}%
  \def\examplelines{}%
  \readexamplelines
  \closein\examplesource
  \expandafter\endofcodeexample\expandafter{\examplelines}%
}

\def\readexamplelines{
  \ifeof\examplesource%
  \else
    \immediate\read\examplesource to \exampleline
    \expandafter\expandafter\expandafter\def\expandafter\expandafter\expandafter\examplelines\expandafter\expandafter\expandafter{\expandafter\examplelines\exampleline}
    \expandafter\readexamplelines%
  \fi
}

\let\code@pre\pgfutil@empty
\let\code@post\pgfutil@empty
\let\code@render\pgfutil@empty
\def\code@catcode@hook{}

\newif\ifcodeexamplefromfile
\newdimen\codeexamplewidth
\newif\ifcode@execute
\newbox\codeexamplebox
\def\codeexample[#1]{%
  \begingroup%
  \code@executetrue
  \pgfqkeys{/codeexample}{every codeexample,#1}%
  \ifpgfmanualtikzsyntaxhilighting%
  \pgfkeys{/codeexample/syntax hilighting}%
  \fi%
  \ifcodeexamplefromfile\begingroup\fi
  \codeexample@tabinit% assigns \codeexample@tabinit@[catcode,replacementchars]
  \parindent0pt
  \begingroup%
  \par%
  \medskip%
  \let\do\@makeother%
  \dospecials%
  \obeylines%
  \@vobeyspaces%
  \catcode`\%=13%
  \catcode`\^^M=13%
  \code@catcode@hook%
  \codeexample@tabinit@catcode
  \relax%
  \ifcodeexamplefromfile%
    \expandafter\do@codeexamplefromfile%
  \else%
    \expandafter\find@example%
  \fi}
\def\endofcodeexample#1{%
  \endgroup%
  \ifcode@execute%
    \setbox\codeexamplebox=\hbox{%
      \ifx\code@render\pgfutil@empty%
      {%
        {%
          \returntospace%
          \commenthandler%
          \xdef\code@temp{#1}% removes returns and comments
        }%
        \catcode`\^^M=9%
        \colorbox{graphicbackground}{\color{black}\ignorespaces%
          \code@pre\expandafter\scantokens\expandafter{\code@temp\ignorespaces}\code@post\ignorespaces}%
      }%
      \else%
        \colorbox{graphicbackground}{\color{black}\ignorespaces%
          \code@render}%
      \fi%
    }%
    \ifdim\wd\codeexamplebox>\codeexamplewidth%
      \def\code@start{\par}%
      \def\code@flushstart{}\def\code@flushend{}%
      \def\code@mid{\parskip2pt\par\noindent}%
      \def\code@width{\linewidth-6pt}%
      \def\code@end{}%
    \else%
      \def\code@start{%
        \linewidth=\textwidth%
        \parshape \@ne 0pt \linewidth
        \leavevmode%
        \hbox\bgroup}%
      \def\code@flushstart{\hfill}%
      \def\code@flushend{\hbox{}}%
      \def\code@mid{\hskip6pt}%
      \def\code@width{\linewidth-12pt-\codeexamplewidth}%
      \def\code@end{\egroup}%
    \fi%
    \code@start%
    \noindent%
    \begin{minipage}[t]{\codeexamplewidth}\raggedright
      \hrule width0pt%
      \footnotesize\vskip-1em%
      \code@flushstart\box\codeexamplebox\code@flushend%
      \vskip-1ex
      \leavevmode%
    \end{minipage}%
  \else%
    \def\code@mid{\par}
    \def\code@width{\linewidth-6pt}
    \def\code@end{}
  \fi%
  \code@mid%  
  \colorbox{codebackground}{%
    \pgfkeysvalueof{/codeexample/prettyprint/base color}%
    \begin{minipage}[t]{\code@width}%
      {%
        \let\do\@makeother
        \dospecials
        \frenchspacing\@vobeyspaces
        \normalfont\ttfamily\footnotesize
        \typesetcomment%
        \codeexample@tabinit@replacementchars
        \@tempswafalse
        \def\par{%
          \if@tempswa
          \leavevmode \null \@@par\penalty\interlinepenalty
          \else
          \@tempswatrue
          \ifhmode\@@par\penalty\interlinepenalty\fi
          \fi}%
        \obeylines
        \everypar \expandafter{\the\everypar \unpenalty}%
        \pgfkeysvalueof{/codeexample/typeset listing/.@cmd}{#1}\pgfeov
      }
    \end{minipage}}%
  \code@end%
  \par%
  \medskip
  \endcodeexample\endgroup
}

\def\endcodeexample{\endgroup}


\makeatother

\usepackage{pgfmanual}


% autoxref is now always on

% \makeatletter
% % \pgfautoxrefs will be defined by 'make dist'
% \pgfutil@ifundefined{pgfautoxrefs}{%
%   \renewcommand\pgfmanualpdflabel[3][]{#3}% NO-OP
%   \def\pgfmanualpdfref#1#2{#2}%
%   \pgfkeys{
%     /pdflinks/codeexample links=false,% DISABLED.
%   }%
% }{}
% \makeatother

\newdimen\pgfmanualcslinkpreskip

% Styling of the pretty printer
\pgfkeys{
  /codeexample/syntax hilighting/.style={
    /codeexample/prettyprint/key name/.code={\textcolor{green!50!black}{\pgfmanualpdfref{##1}{##1}}},
    /codeexample/prettyprint/key name with handler/.code 2 args={\textcolor{green!50!black}{\pgfmanualpdfref{##1}{##1}}/\textcolor{blue!70!black}{\pgfmanualpdfref{/handlers/##2}{##2}}},
    /codeexample/prettyprint/key value display only/.code={\textcolor{green!50!black}{{\itshape{\let\pgfmanualwordstartup\relax\pgfmanualprettyprintcode{##1}}}}},
    /codeexample/prettyprint/cs/.code={\textcolor{blue!70!black}{\pgfmanualcslinkpreskip4.25pt\pgfmanualpdfref{##1}{##1}}},
    /codeexample/prettyprint/cs with args/.code 2 args={\textcolor{black}{\pgfmanualcslinkpreskip4.25pt\pgfmanualpdfref{##1}{##1}}\{\textcolor{black}{\pgfmanualprettyprintcode{##2}}\pgfmanualclosebrace},
    /codeexample/prettyprint/cs arguments/pgfkeys/.initial=1,
    /codeexample/prettyprint/cs/pgfkeys/.code 2 args={\textcolor{black}{\pgfmanualcslinkpreskip4.25pt\pgfmanualpdfref{##1}{##1}}\{\textcolor{black}{\pgfmanualprettyprintpgfkeys{##2}}\pgfmanualclosebrace},
    /codeexample/prettyprint/cs arguments/begin/.initial=1,
    /codeexample/prettyprint/cs/begin/.code 2 args={\textcolor{black}{##1}\{\textcolor{blue!70!black}{\pgfmanualpdfref{##2}{##2}}\pgfmanualclosebrace},
    /codeexample/prettyprint/cs arguments/end/.initial=1,
    /codeexample/prettyprint/cs/end/.code 2 args={\textcolor{black}{##1}\{\textcolor{blue!70!black}{\pgfmanualpdfref{##2}{##2}}\pgfmanualclosebrace},
    /codeexample/prettyprint/word/.code={\pgfmanualwordstartup{\begingroup\pgfkeyssetvalue{/pdflinks/search key prefixes in}{}\pgfmanualpdfref{##1}{##1}\endgroup}},
    /codeexample/prettyprint/point/.code={\textcolor{violet}{##1}},%
    /codeexample/prettyprint/point with cs/.code 2 args={\textcolor{violet}{(\pgfmanualpdfref{##1}{##1}:##2}},%
    /codeexample/prettyprint/comment font=\itshape,
    /codeexample/prettyprint/base color/.initial=\color{black!55},
    /pdflinks/render hyperlink/.code={%
      {\setbox0=\hbox{##1}%
        \rlap{{\color{white}\dimen0\wd0\advance\dimen0by-\pgfmanualcslinkpreskip\hskip\pgfmanualcslinkpreskip\vrule width\dimen0 height-1pt depth1.6pt}}%
        \box0%
      }%
    }
  }  
}

\def\pgfmanualwordstartup{\textcolor{black}}


%%% Local Variables: 
%%% mode: latex
%%% TeX-master: "beameruserguide"
%%% End: 

\usepackage{makeidx}
\usepackage{hyperref}

\title{\texttt{pktxdiag.sty}\\ A TikZ-based \LaTeX{} package\\ for
  drawing packet exchange diagrams}
\author{Christophe Deleuze}
\date{Draft user guide - \today}

\makeindex
\begin{document}
\maketitle
\hypersetup{colorlinks=true, linkcolor=black}
\tableofcontents
\newpage
\setlength{\parindent}{0mm}
\setlength{\parskip}{1mm}

% for pgfmanual-en-macros
\pgfkeys{/pktxd/background=graphicbackground}

%\newcommand{\options}[1][options]{\opt{\oarg{#1}}}

% undocumented:
% ABtimed, ABtimedi, ABcancel, ABcanceli, ABdelayed, ABstop (and BA of course)
% settimerA, settimerAi, canceltimerA, canceltimerAi

% env: reversed, width (mentioned only for multi-mode)

\section{Using the package}

If you install the package manually (from eg
\url{https://gitlab.com/deleuzec/pktxdiag.sty}) simply ensure the
file \texttt{pktxdiag.sty} is in \LaTeX's load path.  A quick way to
give it a try is to copy it in the directory from which you run the
compilation commands.  For a more robust way, copy the file to a
proper location (see section 3.4.6 of the texlive guide for example).

Load the package with \texttt{usepackage}:

\begin{verbatim}
  \documentclass{article}
  \usepackage{pktxdiag}
  \begin{document}
  ...
  \end{document}
\end{verbatim}

\section{Basic diagrams}

\subsection{Simple exchanges}

A single DATA frame, acknowledged by an ACK frame.  A and B stand for
the left and right \emph{entities} (ie what sends and receives
frames).
\begin{codeexample}[width=6cm]
  \begin{pktxdiag}
    \AB{DATA}
    \BA{ACK}
  \end{pktxdiag}
\end{codeexample}

A frame can be corrupted in transit...
\begin{codeexample}[width=6cm]
  \begin{pktxdiag}
    \AB[bad]{DATA}
    \BA{NAK}
  \end{pktxdiag}
\end{codeexample}
\index{frame option!bad}

\subsection{Timers}

Frame loss is usally dealt with through a retransmission timer.
\begin{codeexample}[width=6cm]
  \begin{pktxdiag}
    \AB[timed={3cm}{retrans timer},lost]{DATA}
    \Awaittimer    
    \AB{DATA}
  \end{pktxdiag}
\end{codeexample}
\index{frame option!timed}
\index{frame option!lost}
\index{command!Awaittimer}
\index{command!Bwaittimer}

Such a timer is cancelled when an ACK frame is received.
\begin{codeexample}[width=6cm]
  \begin{pktxdiag}
    \AB[timed={3cm}{retrans timer}]{DATA}
    \BA[cancel]{ACK}
  \end{pktxdiag}
\end{codeexample}
\index{frame option!cancel}

% \begin{codeexample}[width=6cm]
%   \begin{pktxdiag}
%   \AB[timed={3cm}{temporisateur}]{DATA}
%   \BA[lost]{ACK}
%   \Awaittimer
%   \AB[timed={3cm}{temporisateur}]{DATA}
%   \BA[cancel]{ACK}
% \end{pktxdiag}
% \end{codeexample}
% \index{frame option!lost}

ACKs are often \emph{delayed}: they are not sent immediately but after
an \emph{ack timer} expires:
\begin{codeexample}[width=6cm]
  \begin{pktxdiag}
    \AB{DATA 2,x}
    \BA[delayed={1cm}{ack}]{ACK 3}
  \end{pktxdiag}
\end{codeexample}

The purpose of \emph{delayed ACKs} is that sending of a dedicated ACK
frame can be cancelled if a DATA frame is to be sent shortly: the ACK
will be \emph{piggybacked} on the data frame.
\begin{codeexample}[width=6cm]
 \begin{pktxdiag}
   \AB{DATA 2,x}
   \BA[stop={1cm}{ack}{3mm}]{DATA y,3}
 \end{pktxdiag}
\end{codeexample}

\index{frame option!delayed}
\index{frame option!stop}

Currently four timers may run simultaneously on each side.  This may
be increased in future versions.

\index{frame option!timed2}
\index{frame option!delayed2}
\index{frame option!stop2}
\index{frame option!cancel2}
\index{frame option!timed3}
\index{frame option!delayed3}
\index{frame option!stop3}
\index{frame option!cancel3}
\index{frame option!timed4}
\index{frame option!delayed4}
\index{frame option!stop4}
\index{frame option!cancel4}

\begin{codeexample}[width=6cm]
 \begin{pktxdiag}
   \AB[timed={3cm}{1st timer}]{DATA 1}
   \BA[stop2={1cm}{ack}{5mm},timed={3cm}{timed},cancel]{DATA}
   \AB[timed2={2cm}{2nd timer},lost]{DATA 2}
   \Awaittimer[2]
   \AB[timed2={2cm}{2nd timer}]{DATA 2}
   \BA[cancel2]{ACK}
 \end{pktxdiag}
\end{codeexample}

\subsection{Bad timers}

When a frame holds the \texttt{cancel} option, the corresponding timer
must be running when the frame is received.  If this is not the case,
the compilation fails with an error signaling the problem.  However,
if the option \texttt{warn only} is passed to the environment (or the
option \texttt{warnonly}\footnote{Unfortunately, package options can't
  contain space characters.} to the package) only a warning is issued.
In either case a prominent yellow box is drawn at the place the
problem occurs to help locate and correct the problem.

\index{package option!warnonly}
\index{env option!warn only}

% TODO cancel before any timer was run raises a latex error

\begin{codeexample}[width=7cm]
\begin{pktxdiag}[warn only]
  \AB[timed={1cm}{timer}]{DATA}
  \BA[cancel]{ACK}
\end{pktxdiag}
\end{codeexample}

\begin{codeexample}[width=7cm]
\begin{pktxdiag}[warn only]
  \BA[timed={2cm}{timer}]{DATA}
  \AB[cancel]{ACK}
  \AB[cancel]{ACK}
\end{pktxdiag}
\end{codeexample}

\subsection{Late frames}

Sometimes a frame is delayed in the network and arrives \emph{late}.
\index{command!Awaitlate}
\index{command!Bwaitlate}

\begin{codeexample}[width=6cm]
  \begin{pktxdiag}
    \AB[timed={2cm}{retrans}]{DATA 1}
    \BA[cancel]{ACK}
    \AB[late=4cm,timed2={2cm}{retrans}]{DATA 2}
    \Awaittimer[2]
    \AB[timed={2cm}{retrans}]{DATA 2}
    \BA[cancel]{ACK}
    \Bwaitlate
    \BA{ACK}
  \end{pktxdiag}
\end{codeexample}

If several late frames are used, they can be named to be waited upon.
The unamed late frame is actually named \texttt{\$\$lateframe}.
\index{frame option!name}

\begin{codeexample}[width=6cm]
\begin{pktxdiag}
  \AB{DATA1}
  \AB[late=2cm]{DATA2}
  \AB[late,name=n2]{DATA3}
  \Bwaitlate
  \BA{ACK3}
  \Bwaitlate[n2]
  \BA{ACK4}
\end{pktxdiag}
\end{codeexample}

\subsection{Gone and plop frames}

A lost frame will never arrive.  We define \emph{gone} frames as just disappearing: they're sent and we don't know (or care) what happens next.  A \emph{plop} frame is the reverse: a frame appears and is received, but we don't know when/if it was sent.

\index{frame option!gone}\index{frame option!plop}

\begin{codeexample}[width=6cm]
\begin{pktxdiag}
  \BA[gone]{ACK}
  \BA[plop]{ACK}
\end{pktxdiag}
\end{codeexample}

\subsection{Time is passing}

\begin{description}
\item[\texttt{prop}] The propagation delay is the time between the
  sending and the receiving of the frame.  One can obtain horizontal
  arrows by setting it to 0mm.

\item[\texttt{send delay}] The time between the transmission of two
  consecutive frames by the same sender.

\item[\texttt{reply delay}] The time between the receipt of a frame
  and the sending of an answer.
\end{description}

\index{env option!prop}
\index{env option!send delay}
\index{env option!reply delay}

Commands \texttt{Apause} and \texttt{Bpause} make the entity pause
some amount of time before sending again.  The default is 1cm and can
be changed with the \texttt{pause} environment option.

\index{command!Apause}
\index{command!Bpause}
\index{env option!pause}

\begin{codeexample}[]
\begin{pktxdiag}[send delay=6mm,reply delay=0mm]
  \AB{DATA}
  \AB{DATA}
  \BA{ACK}
  \Apause
  \AB{DATA}
  \Bpause[5mm]
  \BA{ACK}
\end{pktxdiag}
\end{codeexample}

The \texttt{later} command allows time to ``skip''.  Time lines
(including timers) are dashed.  However, late frames are not (this can
be considered a bug).

\index{command!later}
% TODO should have default value?

\begin{codeexample}[width=8.5cm]
\begin{pktxdiag}
  \AB{DATA1}
  \AB[timed={6cm}{timer}]{DATA2}
  \AB[timed2={6cm}{timer2}]{DATA3}
  \later{2cm}
  \BA[cancel]{ACK}
  \BA[cancel2]{ACK}
\end{pktxdiag}
\end{codeexample}

\subsection{Full duplex exchanges}
\label{sec:full-duplex}

By default, exchanges are in half duplex mode.  When a frame is sent,
the next sending time is advanced by \texttt{send delay}.  If a frame
is then received, this time is moved to the reception time plus
\texttt{reply delay}.

The \texttt{full duplex} environment option sets full duplex mode,
where the second rule does not apply.  While in full duplex mode,
commands \texttt{AwaitB} and \texttt{BwaitA} make one entity
synchronize to the other (that is, apply the second rule above
punctually).  Commands \texttt{fullduplex} and \texttt{halfduplex}
allow switching between the two modes.

\index{env option!full duplex}
\index{command!AwaitB}
\index{command!BwaitA}
\index{command!fullduplex}
\index{command!halfduplex}

When in full duplex mode, labels are drawn close to the sender instead
of being centered on the arrow.  See also section \ref{sec:labels} on
label styles.

\begin{codeexample}[width=8cm]
  \begin{pktxdiag}[width=5cm]
    \BA{ACK 0}
    \AB[timed={2cm}{retrans}]{DATA 0}
    \BA[late=1cm]{ACK 1}
    \Awaittimer\AB{DATA 0}
    \fullduplex
                \BA{ACK 1}
    \Awaitlate
    \AB{DATA 1}
                \BA{ACK 0}
    \AB{DATA 1}
                \BA{ACK 0}
    \AB{DATA 0}
                \BA{ACK 1}
    \AB{DATA 0}
                \BA{ACK 1}
  \end{pktxdiag}
\end{codeexample}

%fullduplex before the ACK that shouldn't synchronize the other

\section{Decorations}

\subsection{Entities names and pictures}

Entities names can be provided with the \texttt{A} and \texttt{B}
keys.

\index{env option!A}
\index{env option!B}

\begin{minipage}{.5\linewidth}
\begin{codeexample}[]
  \begin{pktxdiag}[A,B]
    \AB{to B}
    \BA{from B}
  \end{pktxdiag}
\end{codeexample}
\end{minipage}
\begin{minipage}{.5\linewidth}
\begin{codeexample}[]
  \begin{pktxdiag}[A=client,B=server]
    \AB{request}
    \BA{reply}
  \end{pktxdiag}
\end{codeexample}
\end{minipage}

A picture can also be set with \texttt{Apic}/\texttt{Bpic}.  It uses
the \texttt{graphicx} package.  Arguments will be provided to
\verb+\includegraphics+ (first argument is the file name, second is
the list of options).

By default, the name (if provided) is above the picture, key
\texttt{pic on top} reverses that. \texttt{pic offset} sets the
vertical space between name and picture.

\index{env option!Apic}
\index{env option!Bpic}
\index{env option!pic on top}
\index{env option!pic offset}

\begin{minipage}{.5\linewidth}
\begin{codeexample}[]
\begin{pktxdiag}[
  A=laptop, Apic={laptop}{height=1cm},
  B=fileserver, Bpic={fileserver}{height=1cm},
  pic offset=2mm]
    \AB{request}
    \BA{reply}
  \end{pktxdiag}
\end{codeexample}
\end{minipage}
\begin{minipage}{.5\linewidth}
\begin{codeexample}[]
\begin{pktxdiag}[
  A=laptop, Apic={laptop}{height=1cm},
  B=fileserver, Bpic={fileserver}{height=1cm},
  pic on top]
    \AB{request}
    \BA{reply}
  \end{pktxdiag}
\end{codeexample}
\end{minipage}

\subsection{Lost frames}

A cross marks the disappearing of lost frames.  The cross can be removed.

\index{frame option!lost}
\index{frame option!no cross}
\index{env option!no cross}

\begin{codeexample}[]
  \begin{pktxdiag}
    \AB[lost]{DATA 0} \AB[lost]{DATA 1}
  \end{pktxdiag}
~~~~~~~~
  \begin{pktxdiag}[]
    \AB[lost]{DATA 0} \AB[lost,no cross]{DATA 1}
  \end{pktxdiag}
~~~~~~~~
  \begin{pktxdiag}[no cross]
    \AB[lost]{DATA 0} \AB[lost,no cross=false]{DATA 1}
  \end{pktxdiag}
\end{codeexample}

\subsection{Timers}

\subsubsection{Timer offsets}

The distance between the entity time line and the timer line/text
can be set with the environment options \texttt{timer offset} and
\texttt{timerX offset} with \texttt{X} being \texttt{2}, \texttt{3},
or \texttt{4}.  This can be set for only A or B with \texttt{A timer
  offset}, \texttt{B timer2 offset} etc.

One can also set the first timer offset with \texttt{timer offset} and
the delta for the other timers with \texttt{timers/+offset}.  The key
\texttt{timers/tight} is a shortcut to set both at 2mm (which is ok if
no labels are put on timers).

\begin{codeexample}[]
\begin{pktxdiag}[timer offset=3mm,timers/+offset=3mm]
  \fullduplex
  \AB[timed={4cm}{data 0}]{DATA 0}
  \BwaitA
  \BA[delayed={6mm}{ack},cancel]{ACK 1}
  \AB[timed2={4cm}{data 1}]{DATA 1}
  \BA[delayed={6mm}{ack},cancel2]{ACK 2}
  \AB[timed3={4cm}{data 2}]{DATA 2}
  \BA[delayed={6mm}{ack},cancel3]{ACK 3}
  \AB[timed4={4cm}{data 3}]{DATA 3}
  \BA[delayed={6mm}{ack},cancel4]{ACK 4}
\end{pktxdiag}

\begin{pktxdiag}[timers/tight,reversed]
  \fullduplex
  \AB[timed={4cm}{},lost,label style=midway]{DATA 0}
  \AB[timed2={4cm}{}]{DATA 1}
  \BwaitA
  \BA[delayed={3mm}{ack}]{ACK 0}
  \AB[timed3={4cm}{}]{DATA 2}
  \BA[delayed={3mm}{ack}]{ACK 0}
  \AB[timed4={4cm}{}]{DATA 3}
  \BA[delayed={3mm}{ack}]{ACK 0}
  \Awaittimer[1]
  \AB{DATA 0}
  \BwaitA
  \BA[cancel3,cancel4]{ACK 4}
  \AB{DATA 1}
\end{pktxdiag}
\end{codeexample}

\index{env option!timer offset}
\index{env option!timer2 offset}
\index{env option!timer3 offset}
\index{env option!timer4 offset}
\index{env option!A timer offset}
\index{env option!A timer2 offset}
\index{env option!A timer3 offset}
\index{env option!A timer4 offset}
\index{env option!B timer offset}
\index{env option!B timer2 offset}
\index{env option!B timer3 offset}
\index{env option!B timer4 offset}
\index{env option!timers/+offset}
\index{env option!timers/tight}


% TODO also A/B specific offsets

% TODO can be used for a specific frame? maybe not

% TODO timer time offset (space arrows of consecutive timers)

\subsubsection{Timer label on arrow}

By default, timer labels are written ``above'' the arrow, but can be
written on the arrow if the label is short enough.  By default, the
label must be 5mm shorter than the arrow.

\begin{codeexample}[]
\def\trace{
    \AB[timed={3cm}{timer for 0}]{DATA 0,0}
    \AB[timed2={3cm}{the timer for 1}]{DATA 1,0}
    \BA[stop={2cm}{ack}{1cm}]{DATA 0,2}}
  \begin{pktxdiag}
    \trace
  \end{pktxdiag}
  \begin{pktxdiag}[timers/on arrow]
    \trace
  \end{pktxdiag}
  \begin{pktxdiag}[timers/on arrow=1cm]
    \trace
  \end{pktxdiag}
\end{codeexample}

\index{env option!timers/on arrow}

\subsubsection{Left timer label positions}

If the previous option is not set, there are a few more choices for
the position of a left timer label.  The default setting (left
picture) is probably what you want...

\index{env option!timers/left below}
\index{env option!timers/left bottom up}

\begin{codeexample}[]
\def\trace{
    \AB[timed={3cm}{timer for 0}]{DATA 0,0}
    \AB[timed2={3cm}{timer for 1}]{DATA 1,0}
    \BA[stop={2cm}{ack}{1cm}]{DATA 0,2}}
  \begin{pktxdiag}
    \trace
  \end{pktxdiag}
  \begin{pktxdiag}[timers/left below]
    \trace
  \end{pktxdiag}
  \begin{pktxdiag}[timers/left bottom up]
    \trace
  \end{pktxdiag}
\end{codeexample}

\subsection{Send and receive annotations}

Send and receive notes can be associated with frames.  Their
horizontal offset can be adjusted globally in the \texttt{pktxdiag}
environment, or in a specific frame as the example shows.

\index{frame option!snote}
\index{frame option!rnote}
\index{frame option!snoteoffset}
\index{frame option!rnoteoffset}
\index{env option!snote}
\index{env option!rnote}
\index{env option!snoteoffset}
\index{env option!rnoteoffset}

\begin{codeexample}[]
\begin{pktxdiag}
  \AB[snote=send note,rnote=receive note]{DATA}
  \BA[snoteoffset=6mm,rnoteoffset=1mm,snote=less close,rnote=closer]{DATA}
\end{pktxdiag}
\end{codeexample}

\subsection{Labels}
\label{sec:labels}

TODO: frame and env key labelstyle

See also section \ref{sec:flow styles} on flow styles.

In the figure in section \ref{sec:full-duplex} the first few frame
labels where centered (because of half-duplex mode) and the following
ones where near the start of the frame (because of full-duplex mode).
This can be fixed by adding \texttt{near start} to the label style
through the \texttt{label style+} key.

\begin{codeexample}[width=8cm]
  \begin{pktxdiag}[width=5cm,
                   label style+=near start]]
    \BA{ACK 0}
    \AB[timed={2cm}{retrans}]{DATA 0}
    \BA[late=1cm]{ACK 1}
    \Awaittimer\AB{DATA 0}
    \fullduplex
                \BA{ACK 1}
    \Awaitlate
    \AB{DATA 1}
                \BA{ACK 0}
    \AB{DATA 1}
                \BA{ACK 0}
    \AB{DATA 0}
                \BA{ACK 1}
    \AB{DATA 0}
                \BA{ACK 1}
  \end{pktxdiag}
\end{codeexample}


% TODO env option show label2
% env option hide label
% frame option label2

\subsection{Graphics attributes}

Frames can also be given most tikz graphics attributes to specify their
colors, line width, arrow tip etc.

% arrow tip default is -Latex
% show possibilities ->, -latex, -Stealth (could be default?), -Triangle

\begin{codeexample}[width=5cm]
  \begin{pktxdiag}
    \AB[,{-Latex[open]}]{DATA}
    \BA[dashed,blue,thin,->]{ACK}
    \AB[ultra thick,orange,text=red,-Stealth]{OK !}
  \end{pktxdiag}
\end{codeexample}

% TODO: how to change label graphics attributes?

The background color (default \texttt{white}, but not in this very
document) can be set with the \texttt{background} env
option.\index{env option!background}

\begin{codeexample}[width=5cm]
  \begin{pktxdiag}[background=green!15]
    \AB[thin,-Straight Barb]{DATA}
    \BA[dashed,blue,-latex]{ACK}
    \AB[ultra thick,orange,text=red,
        -{Stealth[length=2.5mm,inset=0pt]}]{OK !}
  \end{pktxdiag}
\end{codeexample}




\subsection{Actions and states}

Actions and states are another kind of annotations, not directly
related to the sending or receiving of a frame.  Here's an example
showing the application actions on the TCP layer and the states of the
TCP state machine.

\index{command!Astate}
\index{command!Bstate}
\index{command!Aactionstate}
\index{command!Bactionstate}

\texttt{action offset} and \texttt{state offset} set the distance
between the timeline and the action/state. \texttt{action time offset}
offsets the action in the past so that there's some delay between the
action and any frame send that occurs after it.

\index{env option!action time offset}
\index{env option!action offset}
\index{env option!state offset}

\begin{codeexample}[]
\begin{pktxdiag}[prop=7mm,A=client,B=server,action time
  offset=2mm,action offset=2mm]
  \Astate{Closed}\Bstate{Closed}
  \Apause         \Bpause
  \Apause         \Bactionstate{passive open}{Listen}
  \Aactionstate{active open}{Syn-Sent}
  \AB{SYN}
                 \Bstate{SynRecvd}
                 \BA{SYN+ACK}
  \Astate{Established}
  \AB{ACK}
                 \Bstate{Established}
\end{pktxdiag}
\end{codeexample}

% TODO - show action time offset

\section{Flows}

\subsection{Flows}

Each frame belongs to a given \emph{flow}, as specified by the
\texttt{fid} option\index{frame option!fid}.  A frame with no
\texttt{fid} option belongs to the \texttt{main} flow.

\subsection{Filtering}

When a diagram contains two (or more) flows, one may want to make one
less visible to draw the attention on the other.  A flow can be made
less visible by:

\begin{itemize}
\item \emph{fading} it, through the \texttt{fade} and \texttt{fade
    value} environment options,
\item hiding it, through the \texttt{hide} environment option,
\item ignore it completely, through the \texttt{ignore} environment
  option.
\end{itemize}

\index{env option!fade}
\index{env option!hide}
\index{env option!ignore}
\index{env option!fade value}

% TODO: ho my, an X25 example, really nothing more current?
\begin{codeexample}[]
\def\testfilter{
  \AB[fid=1]{DATA 1 (1)}
  \AB[fid=2]{DATA 1 (2)}
  \AB[fid=2]{DATA 2 (2)}
                      \BA[fid=2]{RR 3 (2)}
  \AB[fid=1]{DATA 2 (1)}
                      \BA[fid=1]{RR 3 (1)}
}
\tikzset{/pktxd/width=2.15cm}
\begin{pktxdiag}
  \testfilter
\end{pktxdiag}
\begin{pktxdiag}[fade=2,fade=1]
  \testfilter
\end{pktxdiag}
\begin{pktxdiag}[fade=2,fade value=0.05]
  \testfilter
\end{pktxdiag}
\begin{pktxdiag}[hide=2]
  \testfilter
\end{pktxdiag}
\begin{pktxdiag}[ignore=2]
  \testfilter
\end{pktxdiag}
\end{codeexample}

% TODO shallow rather than fade?

The environment option \texttt{fade value} sets how much the flow
remains visible (default is 0.25).  If you need to provide several
arguments to one of these options either repeat the option or use the
\texttt{/.list} key handler as in
\texttt{hide/.list=\{1,2\}}.\footnote{See TikZ \& PGF manual section
  86.4.6 for details.}

\subsection{Flow styles}
\label{sec:flow styles}

The main usage of flows is to apply different sets of graphics
attributes, called styles, to sequences of frames.  Styles can be
applied to flows by commands \texttt{setflowstyle}, that defines the
style of arrows, and \texttt{setflowlabelstyle}, that defines the
style of frame labels.

\index{command!setflowstyle}
\index{command!setflowlabelstyle}
\index{command!definestyle}
\index{command!derivestyle}

One can define a style with the \texttt{definestyle} command, or
inherit an existing style with \texttt{derivestyle}.  The new style
adds its own attributes to those it inherits from its parent style.
The default style for arrows and labels are empty.

% TODO: derivestyle example
% eg, ACKs have a smaller label, near start?

% frame style spills on label style

% \index{default, labeldefault} ?

\begin{codeexample}[width=4cm]
\def\testfilter{
  \AB[fid=1]{DATA 1 (1)}
  \AB[fid=2]{DATA 1 (2)}
  \AB[fid=2]{DATA 2 (2)}
                      \BA[fid=2]{RR 3 (2)}
  \AB[fid=1]{DATA 2 (1)}
                      \BA[fid=1]{RR 3 (1)}
}
\definestyle{id1}{green}
\definestyle{id1l}{green!60}
\definestyle{id2}{blue,dashed}
\definestyle{id2l}{blue!60}
\setflowstyle{1}{id1}
\setflowlabelstyle{1}{id1l}
\setflowstyle{2}{id2}
\setflowlabelstyle{2}{id2l}

\begin{pktxdiag}
  \testfilter
\end{pktxdiag}

\begin{pktxdiag}
  \definestyle{1b}{sloped=false,above,draw,orange,fill=orange!20,opacity=0.5}
  \setflowlabelstyle{1}{1b}
  \testfilter
\end{pktxdiag}
\end{codeexample}

TODO interactions between styles and label styles need to be clarified
and possibly cleaned-up

\section{Styles}

\subsection{Frame styles}

% TODO global
% styles/global

% env
% frame style
% frame style+

% frame
% style
% style+ no?

\subsection{Note styles}

\subsection{Entity styles}

Entities are graphically represented as a vertical time line
surrounded by a text label, whose styles are set by \texttt{.style}
environment keys \texttt{entities/style} and \texttt{entities/label
  style}.

\begin{codeexample}[width=5cm]
\begin{pktxdiag}[A,B,
  entities/style/.style={ultra thick, dashed},
  entities/label style/.append style={scale=1.5,orange}]

  \AB{DATA 0}
  \BA{ACK 1}
\end{pktxdiag}
\end{codeexample}

Each entity actually has a style key that initially points to these
keys (which can thus be used to change the default styles), but they
can be redefined to change the style of a given entity.  This can be
usefull in diagrams with multiple entities, see section
\ref{sec:multipleentities}.

\index{env option!entities/style}
\index{env option!entities/label style}

\section{Local and global settings}

TODO

% \pgfkeys{/pktxd/state offset=5mm,/pktxd/timer offset=2mm}


\section{Multiple entities}
\label{sec:multipleentities}

All the previous examples involved only two entities.  Multiple (ie
more than two) entities can be used.

The \texttt{entities} environment key \index{env option!entities}
gives the list of entities, each possibly along with its full name and
distance to its left neighbour.  If not specified, the full name is
the entity id and the distance is taken from \texttt{width} option
\index{env option!width} (or its default value).  The \texttt{width}
key must be included before \texttt{entities} to be effective.

Note that the provided full name can only stand on a single line.  If
such a full name is to be written on more than one line, it must be
defined through the \texttt{entityname}
command. \index{command!entityname}

The command \texttt{frame} is used to send frames.
\index{command!frame} It generalizes the \texttt{AB} and \texttt{BA}
commands by adding two parameters that are the keys of the sending and
receiving entities.

\begin{codeexample}[]
\begin{pktxdiag}[send delay=15mm,reply delay=7mm,prop=0mm,
  entities={C,R//5cm,r/Redirector,D,S/Server S}]
\entityname{C}{Client\\10.0.0.1}
\entityname{R}{Resolver\\20.0.0.1}
\entityname{D}{DNS\\authority}

\definestyle{dns}{blue}
\setflowlabelstyle{dns}{dns}

\frame{C}{r}{GET image}
\frame{r}{C}{302 ip10.0.0.1.example.com/image}
\frame[fid=dns]{C}{R}{ip10.0.0.1.example.com A ?}
\frame[fid=dns,rnote={IP=10.0.0.1\\DNS=20.0.0.1}]{R}{D}{ip10.0.0.1.example.com A ?}
\frame[fid=dns]{D}{R}{ip10.0.0.1.example.com A S}
\frame[fid=dns]{R}{C}{ip10.0.0.1.example.com A S}
\frame{C}{S}{GET ip10.0.0.1.example.com/image}
\frame{S}{C}{200 OK image}
\end{pktxdiag}
\end{codeexample}

Some commands available for two entities diagrams are actually special
versions of a more general command:

\begin{itemize}
\item \texttt{XwaitY} for synchronize: \texttt{AwaitB} is a shorthand
  for \verb+XwaitY{A}{B}+.  Unfortunately, this does not currently
  work as it should: X waits for anyone, not Y specifically.
\item \texttt{Xpause} for pause: \texttt{Apause} is a shorthand for
  \verb+Xpause{A}+.
\end{itemize}

\index{command!XwaitY}
\index{command!Xpause}

Entity pictures can be defined by the \texttt{entitypic}
command. \index{command!entitypic} The following figures also shows
the use of a per entity style.  Note that the per entity style keys
must occur after the \texttt{entities} key (so that the entities exist
at the time the key is processed).

\begin{codeexample}[]
\begin{pktxdiag}[
  entities={L/laptop,R/router,F/firewall,S/fileserver},
  pic offset=1mm,
  reply delay=0mm,
  entities/F/style/.append style={decorate,decoration=zigzag}
  ]
  
  \entitypic{L}{laptop}{height=1cm}
  \entitypic{R}{router}{height=1cm}
  \entitypic{F}{firewall}{height=1cm}
  \entitypic{S}{fileserver}{height=1cm}

  \frame{L}{F}{request file access}
  \frame[dotted,prop=2.5mm]{F}{S}{}
  \frame{F}{L}{reject (unauth)}
\end{pktxdiag}
\end{codeexample}


\begin{codeexample}[]
\begin{pktxdiag}[width=4cm, entities={C/Client, P/Proxy, S/Server},
  % set the default style for entities
  entities/style/.style={ultra thick,blue},
  entities/label style/.append style={scale=1.1,above=2mm,black},
  % set the style for the Proxy entity
  entities/P/style/.style={thin,dashed,dash pattern=on 5pt off 4pt},
  entities/P/label style/.style={scale=0.8, above=0mm,gray}]
  
  \frame{C}{P}{GET index.html}
  \frame{P}{S}{GET index.html}
\end{pktxdiag}
\end{codeexample}

\section{Horizontal diagrams}

There's limited support for drawing horizontal diagrams: simply use
the \texttt{hpktxdiag} environment.  With this kind of diagrams, the
\emph{transmission time}\footnote{That is, the time elapsed between
  the sending of the first and the last bit of the frame.} is shown
and so options are provided to deal with it (they are simply ignored
in vertical diagrams).

\begin{description}
\item[\texttt{trans}] the transmission time for a given frame, or the
  default if given in the environnement.

\index{frame option!trans}
\index{env option!trans}

\item[\texttt{short trans}] the transmission time for ``short
  frames''.  By default such frames are those whose label starts with
  ``ACK''.

\item[\texttt{shorts}] specifies a list of labels: frames carrying one
  of these labels will be short (removes the default ``ACK'' rule).

\item[\texttt{no shorts}] no frames are considered shorts.

\item[\texttt{hdlc}] specifies the list of labels for short frames in
  the HDLC protocol\footnote{What do you mean ``WTF HDLC, OMFG!!''
    ?}. These are RR, RNR, REJ, SREJ, SABM, DISC, UA.

% TODO: replace with shorts=hdlc ?

\end{description}

Timers are supported.  Beware that the transmission time may make that
a timer canceling frame correct in a vertical diagram is incorrect in
a horizontal diagram.

Multiple (more than two) entities are not possible.  Other missing
features include: late frames, laters, gone and plop frames, states,
actions, entity pictures.

\index{env option!short trans}
\index{env option!shorts}
\index{env option!no shorts}
\index{env option!hdlc}

\begin{codeexample}[]
\begin{hpktxdiag}[hdlc,short trans=3mm,A,B,prop=4mm,label style={font=\normalfont}]
  \AB{SABM}
           \BA{UA}
  \fullduplex
  \AB{I 0,0}
             \BwaitA
             \Bpause[4mm]
  \AB{I 1,0}
             \BA{I 0,1}
  \AB{I 2,0}
             \BA{I 1,2}
  \AB{I 3,1}
             \BA{I 2,3}
             \BA{I 3,4}
  \AwaitB
  \Apause
  \AB[gray,text=green]{RR 4}
  \Apause
  \halfduplex
  \AB{DISC}
             \BA{UA}
\end{hpktxdiag}
\end{codeexample}

%\end{document}

\appendix

\section{Gallery}

%Some nice diagrams...

%TODO: ensure each (diagram, source) pair is on same page

\subsection{The need for the TCP \textsc{Time-Wait} state}

\begin{codeexample}[width=8cm]
\begin{pktxdiag}[state offset=5mm,timer offset=2mm,timer2 offset=3mm,width=2cm]
                      \Bactionstate{close}{LastAck}
                      \BAtimed{FIN}{3.5cm}{retrans}
  \Astate{Closed}
  \AB[late=3cm]{ACK}
                      \Bwaittimer
                      \BA[late=6cm,rnote=oops!,rnoteoffset=-1mm]{FIN}
                      \Bstate{Closed}
  \later{2cm}
                      \Bactionstate{passive open}{Listen}
  \Apause
  \Aactionstate{active open}{SynSent}
  \AB{SYN}
                      \Bstate{SynRecvd}
                      \BA{SYN-ACK}
  \Astate{Established}
  \AB{ACK}
                      \Bstate{Established}
  \Awaitlate
  \Astate{CloseWait}
  \AB{ACK}
\end{pktxdiag}
\end{codeexample}

\subsection{IPCP negociation}

\begin{codeexample}[]
\begin{pktxdiag}[A,B,width=2.5cm, fade value=0.1, fade=b]
  \AB[fid=a,snote={IP: 0.0.0.0\\DNS1: 0.0.0.0\\DNS2: 0.0.0.0}]{req id=1}
  \BA[fid=b,snote={IP comp: VJ\\IP: 10.0.0.1}]{req id=1}
  \AB[fid=b,snote=IP comp: VJ]{rej id=1}
  \BA[fid=a,snote={DNS1: 0.0.0.0\\DNS2: 0.0.0.0}]{rej id=1}
  \BA[fid=b,snote=IP: 10.0.0.1]{req id=2}
  \AB[fid=a,snote=IP: 0.0.0.0]{req id=2}
  \AB[fid=b,snote=IP: 10.0.0.1]{ack id=2}
  \BA[fid=a,snote=IP: 10.67.15.36]{nak id=2}
  \AB[fid=a,snote=IP: 10.67.15.36]{req id=3}
  \BA[fid=a,snote=IP: 10.67.15.36]{ack id=3}
\end{pktxdiag}
\end{codeexample}

%% too ugly!
% \subsection{An FTP file transfer}

% \begin{codeexample}[width=6cm]
% \begin{pktxdiag}[width=5cm,prop=2mm]
% \derivestyle{data}{default}{green,line width=2pt}
% \derivestyle{tcp}{default}{gray}
% \setflowstyle{data}{data}
% \derivestyle{tcplabel}{default}
%             {above,blue,font=\footnotesize\ttfamily}
% \setflowlabelstyle{tcp}{tcplabel}
% \setflowstyle{tcp}{tcp}

%   \AB[fid=tcp]{SYN}
%                \BA[fid=tcp]{SYN-ACK}
%   \AB[fid=tcp]{ACK}

%                \BA{220}
%   \AB{USER guest}
%                \BA{331}
%   \AB{PASS xxxx}
%                \BA{230}
%    \AB{SYST}
%                \BA{215 Unix}
%    \AB{PORT 192,168,130,20,217,116}
%                \BA{200}
%    \AB{RETR toto}
%                \BA{150}

%                \BA[fid=data]{SYN}
%    \AB[fid=data]{SYN-ACK}
%                \BA[fid=data]{ACK}
%                \BA[fid=data]{data}
%                \BA[fid=data]{FIN}
%    \AB[fid=data]{ACK}
%    \AB[fid=data]{FIN,ACK}
%                \BA[fid=data]{ACK}

%                \BA{226}
%    \AB{QUIT}
%                \BA{221}

%                \BA[fid=tcp]{FIN}
%    \AB[fid=tcp]{FIN,ACK}
%                \BA[fid=tcp]{ACK}
% \end{pktxdiag}
% \end{codeexample}

\subsection{Time Wait Assassination}
\label{galery:twa}

From rfc1337.

\begin{codeexample}[]
\begin{pktxdiag}[width=6cm]
  \Astate{Established}    \Bstate{Established}
  \Apause
  \Aactionstate{Close}{Fin-Wait-1}
  \AB{seq=100, ack=300, FIN, ACK}
                         \Bstate{Close-Wait}
                         \BA{seq=300, ack=101, ACK}
  \Astate{Fin-Wait-2}
                         \Bpause
                         \Bactionstate{Close}{Last-Ack}
                         \BA{seq=300, ack=101, FIN, ACK}
  \Astate{Time-Wait}
  \AB{seq=101, ack=301, ACK}
                         \Bstate{Closed}

  \later{1cm}
                         \BA[plop,snote=old duplicate]{seq=255, ack=33}
  \AB{seq=101, ack=301, ACK}
                         \BA{seq=301, RST}

  \Astate{Closed}
\end{pktxdiag}
\end{codeexample}


\subsection{Associating an IP address to its DNS resolver}
\label{galery:dnsresolver}

\begin{codeexample}[]
\begin{pktxdiag}[prop=0mm,reply delay=1cm, width=4cm,
    entities={A/client under test,B/recursive resolver,C/myresolver
      auth NS,D/myresolver HTTP server}]
  \entityname{A}{client\\under test}
  \entityname{B}{recursive\\resolver}
  \entityname{C}{myresolver\\authoritative\\name server}
  \entityname{D}{myresolver\\HTTP server}

  \Xpause{A}
  \frame[rnote={Generate a random\\target name}]{A}{D}{GET http://myrevolver.info}
  \frame{D}{A}{307 Redirect Location http://tfme6e1sxk5t.target.myrevolver.info}
  \Xpause{A}
  \frame{A}{B}{DNS Query\\tfme6e1sxk5t.target.myrevolver.info\\IN AAAA / A ?}
  \frame[rnote={Temporarily hold\\the source IP address\\of the
    resolver\\and associate it with\\the target name}]{B}{C}{}
  \Xpause{C}
  \frame{C}{B}{DNS reply\\tfme6e1sxk5t.target.myrevolver.info\\IN AAAA / A HTTP\_SERVER\_ADDRESS}
  \frame{B}{A}{}
  \frame[rnote={Fetch the resolver\\address associated\\with the
    target name}]{A}{D}{\texttt{GET
    http://tfme6e1sxk5t.target.myrevolver.info/}}
  \frame{D}{A}{200 Ok}
\end{pktxdiag}
\end{codeexample}

\subsection{SIP Session}
\label{gallery:sipsession}

\begin{codeexample}[]

\begin{pktxdiag}[width=4cm,reply delay=2mm,send delay=2mm,
  entities={A/bsd-pc.cisco.com,B/cisco.com proxy,
    C/princeton.edu proxy,D/llp-ph.cs.princeton.edu},
  label style+={sloped=false}]

  % latex arrow tip is a bit smaller than Latex
  \definestyle{big}{latex-latex,line width=3mm}
  \setflowstyle{big}{big}

  \frame{A}{B}{invite}
  \frame{B}{C}{invite}
  \frame{C}{D}{invite}
  \frame{B}{A}{100 trying}
  \frame{C}{B}{100 trying}
  \frame{D}{C}{100 trying}
  \Xpause{D}
  \frame{D}{C}{180 ringing}
  \frame{C}{B}{180 ringing}
  \frame{B}{A}{180 ringing}
  \Xpause{D}
  \frame{D}{C}{200 OK}
  \frame{C}{B}{200 OK}
  \frame{B}{A}{200 OK}
  \frame{A}{D}{ACK}
  \Xpause{D}
  \frame[fid=big,prop=0cm]{D}{A}{Media}
  \Xpause{D}
  \frame{D}{A}{BYE}
  \frame{A}{D}{200 OK}
\end{pktxdiag}
\end{codeexample}

\subsection{TODO}

More nice examples should be included!

\section{Producing standalone pictures}

Standalone pictures can be produced with the \texttt{standalone}
document class.  If this is the content of file \texttt{example1.tex}:

\begin{verbatim}
 \documentclass[convert={density=100,outext=.png}]{standalone}
 \usepackage{pktxdiag}
 \begin{document}
   \begin{pktxdiag}
     \AB{DATA}
     \BA{ACK}
   \end{pktxdiag}
 \end{document}
\end{verbatim}

\noindent Running \texttt{pdflatex -shell-escape example1.tex} will
create \texttt{example1.png}.  The conversion to PNG is done with
ImageMagick's \texttt{convert} tool.  Other conversions can be
performed, see the documentation of the \texttt{standalone} document
class.\footnote{Please note that on modern systems, you often need to
  allow for such conversion by editing
  \texttt{/etc/Image-Magick-6/policy.xml} to change
  \verb|<policy domain="coder" rights="none" pattern="PDF" />| to
  \verb|<policy domain="coder" rights="read" pattern="PDF" />|.}

Several pictures can be produced by giving the \texttt{multi} option
to \texttt{standalone}.  File \texttt{example2.tex} below will create
files \texttt{example2-0.png} and \texttt{example2-1.png}.

\begin{verbatim}
\documentclass[convert={density=100},multi={pktxdiag}]{standalone}
\usepackage{pktxdiag}
\begin{document}
   \begin{pktxdiag} % picture 0
     \AB{DATA 0}
     \BA{ACK 1}
   \end{pktxdiag}
   \begin{pktxdiag} % picture 1
     \AB{DATA 1}
     \BA{ACK 0}
   \end{pktxdiag}
 \end{document}
\end{verbatim}

\section{Debugging}

The option \texttt{debug} can be given to either the package or an
environment to get debugging information displayed.

\index{package option!debug}
\index{env option!debug}


\section{The low level API}

TODO

Here we could show the decoupling of the drawing and the protocol
logic (but this is not completely done so far, anyway...)

\section{Credits for examples}

\begin{itemize}
\item \ref{sec:multipleentities} TODO: find it
\item \ref{galery:dnsresolver} TODO: find it
\item \ref{gallery:sipsession} from Peterson \& Davie, ``Computer Networks: A Systems
  Approach'' 3rd edition.  Also see current edition on
  line\footnote{\url{https://book.systemsapproach.org/applications/multimedia.html\#session-control-and-call-control-sdp-sip-h-323}}.
\end{itemize}

\section{Bibliography}

TikZ \& PGF manual


\printindex

\end{document}
