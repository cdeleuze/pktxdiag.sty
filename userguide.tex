\documentclass{ltxdoc}
\usepackage{a4wide}
\usepackage[utf8]{inputenc}
\usepackage{pktxdiag}
\usepackage{pgfmanual}
\usepackage{calc}
\usepackage{subfig}
\input{pgfmanual-en-macros}
\usepackage{makeidx}

\title{\texttt{pktxdiag.sty}\\ A TikZ-based \LaTeX{} package\\ for
  drawing packet exchange diagrams}
\author{Christophe Deleuze}
\date{Draft user guide - \today}

\makeindex
\begin{document}
\maketitle
\tableofcontents
\newpage

%\newcommand{\options}[1][options]{\opt{\oarg{#1}}}

\section{Using the package}

Ensure the file \texttt{pktxdiag.sty} is in \LaTeX's load path.  A fast and easy way is to copy it in the directory from which you run the compilation commands.

You can then simply load the package with \texttt{usepackage}:

\begin{verbatim}
  \documentclass{article}
  \usepackage{pktxdiag}
  \begin{document}
  ...
  \end{document}
\end{verbatim}

\section{Basic diagrams}

\subsection{Simple exchanges}

A single DATA frame, acknowledged by an ACK frame.
\begin{codeexample}[width=6cm]
  \begin{pktxdiag}
    \AB{DATA}
    \BA{ACK}
  \end{pktxdiag}
\end{codeexample}

A frame can be corrupted in transit...
\begin{codeexample}[width=6cm]
  \begin{pktxdiag}
    \AB[bad]{DATA}
    \BA{NAK}
  \end{pktxdiag}
\end{codeexample}
\index{frame option!bad}

\subsection{Timers}

Frame loss is usally dealt with through a retransmission timer.
\begin{codeexample}[width=6cm]
  \begin{pktxdiag}
    \AB[timed={3cm}{retrans timer},lost]{DATA}
    \Awaittimer    
    \AB{DATA}
  \end{pktxdiag}
\end{codeexample}
\index{frame option!timed}
\index{frame option!lost}
\index{command!Awaittimer}
\index{command!Bwaittimer}

Such a timer is cancelled when an ACK frame is received.
\begin{codeexample}[width=6cm]
  \begin{pktxdiag}
    \AB[timed={3cm}{retrans timer}]{DATA}
    \BA[cancel]{ACK}
  \end{pktxdiag}
\end{codeexample}
\index{frame option!cancel}

% \begin{codeexample}[width=6cm]
%   \begin{pktxdiag}
%   \AB[timed={3cm}{temporisateur}]{DATA}
%   \BA[lost]{ACK}
%   \Awaittimer
%   \AB[timed={3cm}{temporisateur}]{DATA}
%   \BA[cancel]{ACK}
% \end{pktxdiag}
% \end{codeexample}
% \index{frame option!lost}

ACKs are often \emph{delayed}: they are not sent immediately but after
an \emph{ack timer} expires:
\begin{codeexample}[width=6cm]
  \begin{pktxdiag}
    \AB{DATA 2,x}
    \BA[delayed={1cm}{ack}]{ACK 3}
  \end{pktxdiag}
\end{codeexample}

The purpose of \emph{delayed ACKs} is that sending of a dedicated ACK
frame can be cancelled if a DATA frame is to be sent shortly: the ACK
will be \emph{piggybacked} on the data frame.
\begin{codeexample}[width=6cm]
  \begin{pktxdiag}
   \AB{DATA 2,x}
   \BA[stop={1cm}{ack}{5mm}]{DATA y,3}
 \end{pktxdiag}
\end{codeexample}

\index{frame option!stop}

Currently only two timers may be run simultaneously on each side.
This may be increased in future versions.

\index{command!Awaittimeri}
\index{command!Bwaittimeri}
\index{frame option!timed2}
\index{frame option!stop2}
\index{frame option!cancel2}

%TODO timed2 but waittimeri

\begin{codeexample}[width=6cm]
  \begin{pktxdiag}
   \AB[timed={3cm}{1st timer}]{DATA 1}
   \BA[stop2={1cm}{ack}{5mm},timed={3cm}{timed},cancel]{DATA}
   \AB[timed2={2cm}{2nd timer},lost]{DATA 2}
   \Awaittimeri
   \AB[timed2={2cm}{2nd timer}]{DATA 2}
   \BA[cancel2]{ACK}
 \end{pktxdiag}
\end{codeexample}

\subsection{Bad timers}

When a frame holds the \texttt{cancel} option, it should arrive before
the timer expires.  If this is not the case, the compilation fails
with an error signaling the problem.  However, if the option
\texttt{warn only} is passed to the environment (or the option
\texttt{warnonly}\footnote{Unfortunately, package options can't
  contain space characters.} to the package) only a warning is issued.
In this case a prominent yellow box is drawn at the place the problem
occurs to help locate and correct the problem.

\index{package option!warnonly}
\index{env option!warn only}

% TODO cancel before any timer was run raises a latex error

\begin{codeexample}[]
\begin{pktxdiag}[warn only]
  \AB[timed={1cm}{timer}]{DATA}
  \BA[cancel]{ACK}
\end{pktxdiag}
\end{codeexample}

\subsection{Late frames}

Sometimes a frame is delayed in the network and arrives \emph{late}.
\index{command!Awaittimeri}
\index{command!Bwaittimeri}
\index{command!Awaitlate}
\index{command!Bwaitlate}

\begin{codeexample}[width=6cm]
  \begin{pktxdiag}
    \AB[timed={2cm}{retrans}]{DATA 1}
    \BA[cancel]{ACK}
    \AB[late=4cm,timed2={2cm}{retrans}]{DATA 2}
    \Awaittimeri
    \AB[timed={2cm}{retrans}]{DATA 2}
    \BA[cancel]{ACK}
    \Bwaitlate
    \BA{ACK}
  \end{pktxdiag}
\end{codeexample}

If several late frames are used, they can be named to be waited upon.
The unamed late frame is actually named \texttt{\$\$lateframe}.
\index{frame option!name}

\begin{codeexample}[]
\begin{pktxdiag}
  \AB{DATA1}
  \AB[late=2cm]{DATA2}
  \AB[late,name=n2]{DATA3}
  \Bwaitlate
  \BA{ACK3}
  \Bwaitlate[n2]
  \BA{ACK4}
\end{pktxdiag}
\end{codeexample}

\subsection{Time is passing}

\begin{description}
\item[\texttt{prop}] The propagation delay is the time between the
  sending and the receiving of the frame.  One can obtain horizontal
  arrows by setting it to 0mm.

\item[\texttt{send delay}] The time between the transmission of two
  consecutive frames by the same sender.

\item[\texttt{reply delay}] The time between the receipt of a frame
  and the sending of an answer.
\end{description}

\index{env option!prop}
\index{env option!send delay}
\index{env option!reply delay}

Commands \texttt{Apause} and \texttt{Bpause} make the entity pause
some amount of time before sending again.  The default is 1cm and can
be changed with the \texttt{pause} environement option.

\index{command!Apause}
\index{command!Bpause}
\index{env option!pause}

\begin{codeexample}[]
\begin{pktxdiag}[send delay=6mm,reply delay=0mm]
  \AB{DATA}
  \AB{DATA}
  \BA{ACK}
  \Apause
  \AB{DATA}
  \Bpause[5mm]
  \BA{ACK}
\end{pktxdiag}
\end{codeexample}

The \texttt{later} command allows time to ``skip''.  Time lines
(including timers) are dashed.  However, late frames are not (this can
be considered a bug).

\index{command!later}
% TODO should have default value?

\begin{codeexample}[width=6cm]
\begin{pktxdiag}
  \AB{DATA1}
  \AB[timed={6cm}{timer}]{DATA2}
  \AB[timed2={6cm}{timer2}]{DATA3}
  \later{2cm}
  \BA[cancel]{ACK}
  \BA[cancel2]{ACK}
\end{pktxdiag}
\end{codeexample}

\subsection{Full duplex exchanges}

By default, exchanges are in half duplex mode.  When a frame is sent,
the next sending time is advanced by \texttt{send delay}.  If a frame
is then received, this time is moved to the reception time plus
\texttt{reply delay}.

The \texttt{full duplex} environment option sets full duplex mode,
where the second rule does not apply.  While in full duplex mode,
commands \texttt{AwaitB} and \texttt{BwaitA} make one entity
synchronize to the other (that is, apply the second rule above
punctually).  Commands \texttt{fullduplex} and \texttt{halfduplex}
allow switching between the two modes.

\index{env option!full duplex}
\index{command!AwaitB}
\index{command!BwaitA}
\index{command!fullduplex}
\index{command!halfduplex}

When in full duplex mode, labels are drawn close to the sender instead
of being centered on the arrow.  See also section \ref{sec:styles} on
label styles.

\begin{codeexample}[width=8cm]
  \begin{pktxdiag}[width=5cm]
    \BA{ACK 0}
    \AB[timed={2cm}{retrans}]{DATA 0}
    \BA[late=1cm]{ACK 1}
    \Awaittimer\AB{DATA 0}
    \fullduplex
    \BA{ACK 1}
    \Awaitlate
    \AB{DATA 1}
    \BA{ACK 0}
    \AB{DATA 1}
    \BA{ACK 0}
    \AB{DATA 0}
    \BA{ACK 1}
    \AB{DATA 0}
    \BA{ACK 1}
  \end{pktxdiag}
\end{codeexample}

%fullduplex before the ACK that shouldn't synchronize the other

\section{Decorations}

\subsection{Entities names}

\index{env option!A}
\index{env option!B}

\begin{minipage}{.6\linewidth}
\begin{codeexample}[]
  \begin{pktxdiag}[A,B]
    \AB{to B}
    \BA{from B}
  \end{pktxdiag}
\end{codeexample}
\end{minipage}
\begin{minipage}{.4\linewidth}
\begin{codeexample}[]
  \begin{pktxdiag}[A=client,B=server]
    \AB{request}
    \BA{reply}
  \end{pktxdiag}
\end{codeexample}
\end{minipage}

\subsection{Lost frames}

By default lost frames are simply shown disappearing in transit.  A
cross can be added to represent their disappearing: 

\index{frame option!lost}
\index{frame option!cross for lost}
\index{env option!cross for lost}

\begin{codeexample}[]
  \begin{pktxdiag}
    \AB[lost]{DATA 0} \AB[lost]{DATA 1}
  \end{pktxdiag}
~~~~~~~~
  \begin{pktxdiag}[cross for lost]
    \AB[lost]{DATA 0} \AB[lost,cross for lost=false]{DATA 1}
  \end{pktxdiag}
~~~~~~~~
  \begin{pktxdiag}
    \AB[lost]{DATA 0} \AB[lost,cross for lost]{DATA 1}
  \end{pktxdiag}
\end{codeexample}

\subsection{Timers}

\subsubsection{Timer offsets}

The distance between the vertical time line and the timer line/text
can be set with the environment options \texttt{timer offset} and
\texttt{timeri offset}.

\index{env option!timer offset}
\index{env option!timeri offset}

% TODO timer2 offset would be better?

% TODO also A/B specific offsets

% TODO can be used for a specific frame? maybe not

\subsubsection{Timer label on arrow}

By default, timer labels are written ``above'' the arrow, but can be
written on the arrow (if the label is short enough).

\begin{codeexample}[]
\def\trace{
    \AB[timed={3cm}{timer for 0}]{DATA 0,0}
    \AB[timed2={3cm}{timer for 1}]{DATA 1,0}
    \BA[stop={2cm}{a very long label for the ack}{1cm}]{DATA 0,2}}
  \begin{pktxdiag}
    \trace
  \end{pktxdiag}
  \begin{pktxdiag}[timers/on arrow]
    \trace
  \end{pktxdiag}
\end{codeexample}

\index{env option!timers/on arrow}

\subsubsection{Left timer label positions}

If the previous option is not set, there are a few more choices for
the position of a left timer label.  The default setting (left
picture) is probably what you want...

\index{env option!timers/left below}
\index{env option!timers/left bottom up}

\begin{codeexample}[]
\def\trace{
    \AB[timed={3cm}{timer for 0}]{DATA 0,0}
    \AB[timed2={3cm}{timer for 1}]{DATA 1,0}
    \BA[stop={2cm}{ack}{1cm}]{DATA 0,2}}
  \begin{pktxdiag}
    \trace
  \end{pktxdiag}
  \begin{pktxdiag}[timers/left below]
    \trace
  \end{pktxdiag}
  \begin{pktxdiag}[timers/left bottom up]
    \trace
  \end{pktxdiag}
\end{codeexample}

\subsection{Send and receive annotations}

Send and receive notes can be associated with frames.  Their
horizontal offset can be adjusted globally in the \texttt{pktxdiag}
environment, or in a specific frame as the example shows.

\index{frame option!snote}
\index{frame option!rnote}
\index{frame option!snoteoffset}
\index{frame option!rnoteoffset}
\index{env option!snote}
\index{env option!rnote}
\index{env option!snoteoffset}
\index{env option!rnoteoffset}

\begin{codeexample}[]
\begin{pktxdiag}
  \AB[snote=send note,rnote=receive note]{DATA}
  \BA[snoteoffset=6mm,rnoteoffset=1mm,snote=less close,rnote=closer]{DATA}
\end{pktxdiag}
\end{codeexample}

\subsection{Labels}

See also section \ref{sec:styles} on styles.

% TODO env option show label2
% env option hide label
% frame option label2

\subsection{Graphics attributes}

Frames can be given most tikz graphics attributes to specify their
colors, line width etc.

\begin{codeexample}[]
  \begin{pktxdiag}
    \AB{DATA}
    \BA[dashed,blue]{ACK}
    \AB[line width=3pt,dotted,yellow,text=red]{OK !}
  \end{pktxdiag}
  \end{codeexample}

% TODO: how to change label graphics attributes?

\subsection{Actions and states}

Actions and states are another kind of annotations, not directly
related to the sending or receiving of a frame.  Here's an example
showing the application actions on the TCP layer and the states of the
TCP state machine.

\index{command!Astate}
\index{command!Bstate}

\texttt{action offset} and \texttt{state offset} set the distance
between the timeline and the action/state. \texttt{action time offset}
offsets the action in the past so that there's some delay between the
action and any frame send that occurs after it.

\index{env option!action time offset}
\index{env option!action offset}
\index{env option!state offset}

\begin{codeexample}[]
\begin{pktxdiag}[prop=7mm,A=client,B=server,action time
  offset=2mm,action offset=2mm]
  \Astate{Closed}\Bstate{Closed}
  \Apause         \Bpause
  \Apause         \Bactionstate{passive open}{listen}
  \Aactionstate{active open}{Syn-Sent}
  \AB{SYN}
                 \Bstate{SynRecvd}
                 \BA{SYN+ACK}
  \Astate{Established}
  \AB{ACK}
                 \Bstate{Established}
\end{pktxdiag}
\end{codeexample}

% TODO - show action time offset

\section{Flows, styles, and filtering}

\subsection{Flows}

Each frame belongs to a given \emph{flow}, as specified by the
\texttt{id} option\index{frame option!id}.  A frame with no
\texttt{id} option belongs to the \texttt{main} flow.

\subsection{Filtering}

When a diagram contains two (or more) flows, one may want to make one
less visible to draw the attention on the other.  A flow can be made
less visible by:

\begin{itemize}
\item \emph{fading} it, through the \texttt{fade} environment option,
\item hiding it, through the \texttt{hide} environment option,
\item ignore it completely, through the \texttt{ignore} environment
  option.
\end{itemize}

\index{env option!fade}
\index{env option!hide}
\index{env option!ignore}
\index{env option!fade value}

\begin{codeexample}[]
\def\testfilter{
  \AB[id=1]{DATA 1 (1)}
  \AB[id=2]{DATA 1 (2)}
  \AB[id=2]{DATA 2 (2)}
                      \BA[id=2]{RR 3 (2)}
  \AB[id=1]{DATA 2 (1)}
                      \BA[id=1]{RR 3 (1)}
}
\tikzset{/pktxd/width=2.15cm}
\begin{pktxdiag}
  \testfilter
\end{pktxdiag}
\begin{pktxdiag}[fade=2,fade=1]
  \testfilter
\end{pktxdiag}
\begin{pktxdiag}[fade=2,fade value=0.05]
  \testfilter
\end{pktxdiag}
\begin{pktxdiag}[hide=2]
  \testfilter
\end{pktxdiag}
\begin{pktxdiag}[ignore=2]
  \testfilter
\end{pktxdiag}
\end{codeexample}

% TODO shallow rather than fade?

The environment option \texttt{fade value} sets how much the flow
remains visible (default is 0.25).  If you need to provide several
arguments to one of these options either repeat the option or use the
\texttt{/.list} key handler as in
\texttt{hide/.list=\{1,2\}}.\footnote{See pgfmanual section 82.4.6 for
  details.}

\subsection{Styles}
\label{sec:styles}

The main usage of flows is to apply different sets of graphics
attributes, called styles, to sequences of frames.  Styles can be
applied to flows by commands \texttt{setflowstyle}, that defines the
style of arrows, and \texttt{setflowlabelstyle}, that defines the
style of frame labels.

\index{command!setflowstyle}
\index{command!setflowlabelstyle}
\index{command!definestyle}
\index{command!derivestyle}

One can define a style with the \texttt{definestyle} command, but it
is often more convenient to use \texttt{derivestyle}.  The new style
adds its own attributes to those it inherits from its parent style.
The default style for arrows is simply called \texttt{default} while
the one for labels is called \texttt{labeldefault}.

% \index{default, labeldefault} ?

\begin{codeexample}[width=4cm]
\def\testfilter{
  \AB[id=1]{DATA 1 (1)}
  \AB[id=2]{DATA 1 (2)}
  \AB[id=2]{DATA 2 (2)}
                      \BA[id=2]{RR 3 (2)}
  \AB[id=1]{DATA 2 (1)}
                      \BA[id=1]{RR 3 (1)}
}
\derivestyle{id1}{default}{green}
\derivestyle{id1l}{labeldefault}{green!60}
\derivestyle{id2}{default}{blue,dashed}
\derivestyle{id2l}{labeldefault}{blue!60}
\setflowstyle{1}{id1}
\setflowlabelstyle{1}{id1l}
\setflowstyle{2}{id2}
\setflowlabelstyle{2}{id2l}

\begin{pktxdiag}
  \testfilter
\end{pktxdiag}

\begin{pktxdiag}
  \definestyle{1b}{above,draw,orange,fill=orange!20,opacity=0.5}
  \setflowlabelstyle{1}{1b}
  \testfilter
\end{pktxdiag}
\end{codeexample}

TODO interactions between styles and label styles need to be clarified
and possibly cleaned-up

\section{Multiple entities}

All the previous examples where conversations involving only two
entities.  Mutiple (more than two) entities can be used.

The \texttt{entities} environment key \index{env option!entities}
gives the list of entities, each possibly along with its full name and
distance to its left neighbour.  If not specified, the full name is
the entity id and the distance is taken from \texttt{width} option
\index{env option!width} (or its default value).

Note that the provided full name can only stand on a single line.  If
such a full name is to be written on more than one line, it must be
defined through the \texttt{entityname}
command. \index{command!entityname}

The command \texttt{frame} is used to send frames.
\index{command!frame} It generalizes the \texttt{AB} and \texttt{BA}
commands by adding two parameters that are the keys of the sending and
receiving entities.

\begin{codeexample}[]
\begin{pktxdiag}[send delay=15mm,reply delay=7mm,prop=0mm,
  entities={C,R//5cm,r/Redirector,D,S/Server S}]
\entityname{C}{Client\\10.0.0.1}
\entityname{R}{Resolver\\20.0.0.1}
\entityname{D}{DNS\\authority}

\derivestyle{dns}{labeldefault}{blue}
\setflowlabelstyle{dns}{dns}

\frame{C}{r}{GET image}
\frame{r}{C}{302 ip10.0.0.1.example.com/image}
\frame[id=dns]{C}{R}{ip10.0.0.1.example.com A ?}
\frame[id=dns,rnote={IP=10.0.0.1\\DNS=20.0.0.1}]{R}{D}{ip10.0.0.1.example.com A ?}
\frame[id=dns]{D}{R}{ip10.0.0.1.example.com A S}
\frame[id=dns]{R}{C}{ip10.0.0.1.example.com A S}
\frame{C}{S}{GET ip10.0.0.1.example.com/image}
\frame{S}{C}{200 OK image}
\end{pktxdiag}
\end{codeexample}

Some commands available for two entities diagram are actually special versions of a more general command:

\begin{itemize}
\item \texttt{XwaitY} for synchronize: \texttt{AwaitB} is a shorthand
  for \verb+XwaitY{A}{B}+.
\item \texttt{Xpause} for pause: \texttt{Apause} is a shorthand for \verb+Xpause{A}+.
\end{itemize}

\index{command!XwaitY}\index{command!Xpause}



\section{Advanced features}

\subsection{Send and receive hooks}

\subsection{Buffer and window management}


\section{Horizontal diagrams}

There's some support for drawing horizontal diagrams: simply use the
\texttt{hpktxdiag} environment.  With this kind of diagrams, the
\emph{transmission time}\footnote{That is, the time elapsed between
  the sending of the first and the last bit of the frame.} is shown
and so options are provided to deal with it (they are simply ignored
in vertical diagrams).

\begin{description}
\item[\texttt{trans}] the transmission time for a given frame, or the
  default if given in the environnement.

\index{frame option!trans}
\index{env option!trans}

\item[\texttt{short trans}] the transmission time for ``short
  frames''.  By default such frames are those whose label starts with ``ACK''.

\item[\texttt{shorts}] specifies a list of labels: frames carrying one
  of these labels will be short (removes the default ``ACK'' rule).

\item[\texttt{no shorts}] no frames are considered shorts.

\item[\texttt{hdlc}] specifies the list of label for short frames in
  the HDLC protocol. There are RR, RNR, REJ, SREJ, SABM, DISC, UA.

% TODO: replace with shorts=hdlc ?

\end{description}

\index{env option!short trans}
\index{env option!shorts}
\index{env option!no shorts}
\index{env option!hdlc}

\begin{codeexample}[]
\begin{hpktxdiag}[hdlc,short trans=3mm,A,B,prop=4mm]
  \AB{SABM}
           \BA{UA}
  \fullduplex
  \AB{I 0,0}
             \BwaitA
             \Bpause[4mm]
  \AB{I 1,0}
             \BA{I 0,1}
  \AB{I 2,0}
             \BA{I 1,2}
  \AB{I 3,1}
             \BA{I 2,3}
             \BA{I 3,4}
  \AwaitB
  \Apause
  \AB{RR 4}
  \Apause
  \halfduplex
  \AB{DISC}
             \BA{UA}
\end{hpktxdiag}
\end{codeexample}

%\end{document}

\appendix

\section{Gallery}

Some nice diagrams...


\subsection{The need for the TCP \textsc{Time-Wait} state}

\begin{codeexample}[width=8cm]
\begin{pktxdiag}[state offset=5mm,timer offset=2mm,timeri offset=3mm,width=2cm]


                      \Bactionstate{close}{LastAck}
                      \BAtimed{FIN}{3.5cm}{retrans}
  \Astate{Closed}
  \AB[late=3cm]{ACK}
                      \Bwaittimer
                      \BA[late=6cm,rnote=teuuh !,rnoteoffset=-1mm]{FIN}
                      \Bstate{Closed}

\later{2cm}
  
                      \Bactionstate{passive open}{Listen}
  \Apause
  \Aactionstate{active open}{SynSent}
  \AB{SYN}
                      \Bstate{SynRecvd}
                      \BA{SYN-ACK}
  \Astate{Established}
  \AB{ACK}
                      \Bstate{Established}
  \Awaitlate
  \Astate{CloseWait}
  \AB{ACK}
\end{pktxdiag}
\end{codeexample}

\subsection{IPCP negociation}

\begin{codeexample}[]
\begin{pktxdiag}[A,B,width=2.5cm, fade value=0.1, fade=b]
  \AB[id=a,snote={IP: 0.0.0.0\\DNS1: 0.0.0.0\\DNS2: 0.0.0.0}]{req id=1}
  \BA[id=b,snote={IP comp: VJ\\IP: 10.0.0.1}]{req id=1}
  \AB[id=b,snote=IP comp: VJ]{rej id=1}
  \BA[id=a,snote={DNS1: 0.0.0.0\\DNS2: 0.0.0.0}]{rej id=1}
  \BA[id=b,snote=IP: 10.0.0.1]{req id=2}
  \AB[id=a,snote=IP: 0.0.0.0]{req id=2}
  \AB[id=b,snote=IP: 10.0.0.1]{ack id=2}
  \BA[id=a,snote=IP: 10.67.15.36]{nak id=2}
  \AB[id=a,snote=IP: 10.67.15.36]{req id=3}
  \BA[id=a,snote=IP: 10.67.15.36]{ack id=3}
\end{pktxdiag}
\end{codeexample}

%% too ugly!
% \subsection{An FTP file transfer}

% \begin{codeexample}[width=6cm]
% \begin{pktxdiag}[width=5cm,prop=2mm]
% \derivestyle{data}{default}{green,line width=2pt}
% \derivestyle{tcp}{default}{gray}
% \setflowstyle{data}{data}
% \derivestyle{tcplabel}{default}
%             {above,blue,font=\footnotesize\ttfamily}
% \setflowlabelstyle{tcp}{tcplabel}
% \setflowstyle{tcp}{tcp}

%   \AB[id=tcp]{SYN}
%                \BA[id=tcp]{SYN-ACK}
%   \AB[id=tcp]{ACK}

%                \BA{220}
%   \AB{USER guest}
%                \BA{331}
%   \AB{PASS xxxx}
%                \BA{230}
%    \AB{SYST}
%                \BA{215 Unix}
%    \AB{PORT 192,168,130,20,217,116}
%                \BA{200}
%    \AB{RETR toto}
%                \BA{150}

%                \BA[id=data]{SYN}
%    \AB[id=data]{SYN-ACK}
%                \BA[id=data]{ACK}
%                \BA[id=data]{data}
%                \BA[id=data]{FIN}
%    \AB[id=data]{ACK}
%    \AB[id=data]{FIN,ACK}
%                \BA[id=data]{ACK}

%                \BA{226}
%    \AB{QUIT}
%                \BA{221}

%                \BA[id=tcp]{FIN}
%    \AB[id=tcp]{FIN,ACK}
%                \BA[id=tcp]{ACK}
% \end{pktxdiag}
% \end{codeexample}

\subsection{Associating an IP address to its DNS resolver}

\begin{codeexample}[]
  \begin{pktxdiag}[prop=0mm,reply delay=1cm, width=4cm,
    entities={A/client under test,B/recursive resolver,C/myresolver
      auth NS,D/myresolver HTTP server}]
  \entityname{A}{client\\under test}
  \entityname{B}{recursive\\resolver}
  \entityname{C}{myresolver\\authoritative\\name server}
  \entityname{D}{myresolver\\HTTP server}

  \Xpause{A}
  \frame[rnote={Generate a random\\target name}]{A}{D}{GET http://myrevolver.info}
  \frame{D}{A}{307 Redirect Location http://tfme6e1sxk5t.target.myrevolver.info}
  \Xpause{A}
  \frame{A}{B}{DNS Query\\tfme6e1sxk5t.target.myrevolver.info\\IN AAAA / A ?}
  \frame[rnote={Temporarily hold\\the source IP address\\of the
    resolver\\and associate it with\\the target name}]{B}{C}{}
  \Xpause{C}
  \frame{C}{B}{DNS reply\\tfme6e1sxk5t.target.myrevolver.info\\IN AAAA / A HTTP\_SERVER\_ADDRESS}
  \frame{B}{A}{}
  \frame[rnote={Fetch the resolver\\address associated\\with the
    target name}]{A}{D}{\texttt{GET
    http://tfme6e1sxk5t.target.myrevolver.info/}}
  \frame{D}{A}{200 Ok}
\end{pktxdiag}
\end{codeexample}

\subsection{SIP Session}

\begin{codeexample}[]
\begin{pktxdiag}[width=4cm,reply delay=2mm,send delay=2mm,
  entities={A/bsd-pc.cisco.com,B/cisco.com proxy,
    C/princeton.edu proxy,D/llp-ph.cs.princeton.edu}]

\derivestyle{big}{default}{latex-latex,line width=3mm}
\derivestyle{def}{labeldefault}{sloped=false}
\setflowstyle{big}{big}
\setflowlabelstyle{main}{def}
\frame{A}{B}{invite}
\frame{B}{C}{invite}
\frame{C}{D}{invite}
\frame{B}{A}{100 trying}
\frame{C}{B}{100 trying}
\frame{D}{C}{100 trying}
\Xpause{D}
\frame{D}{C}{180 ringing}
\frame{C}{B}{180 ringing}
\frame{B}{A}{180 ringing}
\Xpause{D}
\frame{D}{C}{200 OK}
\frame{C}{B}{200 OK}
\frame{B}{A}{200 OK}
\frame{A}{D}{ACK}
\Xpause{D}
\frame[id=big,prop=0cm]{D}{A}{Media}
\Xpause{D}
\frame{D}{A}{BYE}
\frame{A}{D}{200 OK}
\end{pktxdiag}
\end{codeexample}

\subsection{TODO}

More nice examples should be included!

\section{Producing standalone pictures}

Standalone pictures can be produced with the \texttt{standalone} document class.  If this is the content of file \texttt{example1.tex}:

\begin{verbatim}
 \documentclass[convert={density=100,outext=.png}]{standalone}
 \usepackage{pktxdiag}
 \begin{document}
   \begin{pktxdiag}
     \AB{DATA}
     \BA{ACK}
   \end{pktxdiag}
 \end{document}
\end{verbatim}

\noindent running \texttt{pdflatex -shell-escape example1.tex} will create \texttt{example1.png}.  The conversion to PNG is done with ImageMagick's \texttt{convert} tool.  Other conversions can be performed, see the documentation of the \texttt{standalone} document class.

%TODO: also need -alpha remove
%do convert manually

%convert -alpha remove -density 100 file.pdf file.png

\section{The low level API}

TODO

Here we could show the decoupling of the drawing and the protocol
logic (but this is not completely done so far, anyway...)

\printindex

\end{document}
